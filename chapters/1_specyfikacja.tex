\chapter{Specyfikacja tematu projektu}

\section{Cel projektu}

Celem projektu było zaprojektowanie oraz implementacja relacyjnej bazy danych
dla systemu zarządzania magazynem.
Projekt skupia się na stworzeniu spójnej, logicznej oraz znormalizowanej struktury
danych, umożliwiającej efektywne przechowywanie i przetwarzanie informacji
związanych z funkcjonowaniem magazynu.

Głównym założeniem projektu było opracowanie bazy danych, która umożliwia
zarządzanie stanami magazynowymi, produktami, kategoriami, klientami,
pracownikami oraz procesami przyjęcia i wydania towarów.
System wspiera realizację podstawowych operacji na danych (CRUD),
a także zapytań problemowych pozwalających na analizę danych magazynowych.

Celem jest praktyczne zastosowanie
wiedzy z zakresu projektowania relacyjnych baz danych, definiowania relacji
pomiędzy encjami, normalizacji danych oraz integracji bazy danych
z warstwą aplikacyjną.


\section{Zakres projektu}

Zakres projektu obejmuje zaprojektowanie oraz implementację relacyjnej bazy danych
systemu zarządzania magazynem.
Baza danych umożliwia przechowywanie informacji niezbędnych do obsługi
procesów magazynowych oraz zapewnia integralność i spójność danych.

W ramach projektu zrealizowano:
\begin{itemize}
    \item definicję struktur tabel wraz z kluczami głównymi i obcymi,
    \item zaprojektowanie relacji pomiędzy encjami systemu,
    \item implementację operacji CRUD dla kluczowych tabel,
    \item realizację zapytań problemowych o charakterze analitycznym,
    \item opracowanie diagramu związków encji (ERD),
    \item integrację bazy danych z aplikacją posiadającą interfejs graficzny.
\end{itemize}

Projekt obejmuje warstwę bazodanową oraz aplikacyjną.
Interfejs użytkownika umożliwia wygodną obsługę systemu,
natomiast logika biznesowa opiera się na zapytaniach SQL
wykonywanych na relacyjnej bazie danych.


\section{Technologie i narzędzia}

Do realizacji projektu wykorzystano następujące technologie oraz narzędzia:
\begin{itemize}
    \item \textbf{PostgreSQL} – relacyjny system zarządzania bazą danych,
    \item \textbf{pgAdmin4} – narzędzie administracyjne do zarządzania bazą danych,
    \item \textbf{SQL} – język zapytań do definiowania struktury bazy danych
    oraz operacji na danych,
    \item \textbf{JetBrains IntelliJ IDEA} – środowisko programistyczne
    wykorzystane do implementacji aplikacji, logiki systemu oraz interfejsu
    graficznego użytkownika (GUI),
    \item \textbf{Visual Studio Code} – środowisko używane do tworzenia
    dokumentacji projektu w systemie \LaTeX.
    \item \textbf{JDBC} – interfejs komunikacji aplikacji Java z bazą danych,
\end{itemize}


\section{Instrukcja uruchomienia projektu}

Aby poprawnie uruchomić system na lokalnym środowisku, należy postępować
zgodnie z poniższymi krokami:

\begin{enumerate}
    \item \textbf{Pobranie kodu źródłowego:}
    \begin{itemize}
        \item Pobierz projekt z repozytorium GitHub, klikając przycisk \textit{Code},
        a następnie wybierając opcję \textit{Download ZIP}.
        \item Rozpakuj pobrany folder ProjectBazyDanych-2rok-.
        \item Otwórz rozpakowany folder w wybranym środowisku programistycznym
        (zalecane: JetBrains IntelliJ IDEA).
    \end{itemize}

    \item \textbf{Przygotowanie relacyjnej bazy danych (PostgreSQL):}
    \begin{itemize}
        \item Uruchom narzędzie pgAdmin4 i utwórz nową, pustą bazę danych o nazwie SystemZarzadzaniaMagazynem.
        \item Kliknij prawym przyciskiem myszy na utworzoną bazę danych i wybierz
        opcję \textit{Restore}. Wybierz format Plain, i w pole Filename wklej plik "BD\_file.sql" i zatem nacisnij na przycisk \textit{Restore}.
        \item Kliknij prawym przyciskiem myszy na utworzoną bazę danych i wybierz
        opcję \textit{Refresh}.
    \end{itemize}

    \item \textbf{Konfiguracja połączenia z bazą danych:}
    \begin{itemize}
        \item Otwórz projekt aplikacji w środowisku JetBrains IntelliJ IDEA.
        \item Skonfiguruj parametry połączenia z bazą danych w pliku "DataBaseConnection", takie jak:
        nazwa bazy danych, użytkownik oraz hasło.
        \item Upewnij się, że aplikacja posiada poprawny dostęp do serwera
        PostgreSQL.
    \end{itemize}

    \item \textbf{Uruchomienie aplikacji:}
    \begin{itemize}
        \item Upewnij się, że serwer bazy danych PostgreSQL jest uruchomiony.
        \item Uruchom aplikację z poziomu środowiska JetBrains IntelliJ IDEA. Uruchomić plik Main.
        \item Po uruchomieniu aplikacji możliwa jest interakcja z systemem
        poprzez interfejs graficzny użytkownika (GUI).
    \end{itemize}
\end{enumerate}

Po wykonaniu powyższych kroków system zarządzania magazynem jest gotowy
do pracy i umożliwia realizację operacji na danych magazynowych.
