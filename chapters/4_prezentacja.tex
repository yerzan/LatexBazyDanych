\chapter{Przedstawienie powstałej bazy danych i wdrożonych mechanizmów}

\section{Realizacja mechanizmu CRUD}

W systemie zarządzania magazynem zaimplementowano mechanizm CRUD
(\textit{Create, Read, Update, Delete}), który umożliwia wykonywanie
podstawowych operacji na danych przechowywanych w relacyjnej bazie danych
PostgreSQL. Mechanizm ten został zrealizowany przy użyciu procedur oraz
funkcji składowanych napisanych w języku \textit{PL/pgSQL}.

W celu czytelnego zaprezentowania działania mechanizmu CRUD,
jego implementację przedstawiono na przykładzie tabeli
\textbf{clients}, przechowującej dane klientów obsługiwanych przez system.
Dla tej tabeli przygotowano komplet procedur i funkcji umożliwiających
dodawanie, odczyt, aktualizację oraz usuwanie danych.

\subsection{Operacja Create}

Operacja \textit{Create} odpowiada za dodawanie nowych rekordów do tabeli
\texttt{clients}. Zrealizowano ją w postaci procedury składowanej,
która przyjmuje dane klienta jako parametry wejściowe.
Przed zapisaniem danych wykonywana jest walidacja poprawności
najważniejszych pól, takich jak nazwa firmy oraz adres e-mail.
Dzięki temu system zapobiega zapisywaniu niepoprawnych danych w bazie.

\begin{lstlisting}[language=SQL, caption={Procedura dodawania klienta}]
CREATE OR REPLACE PROCEDURE public.clients_create(
    IN p_company_name text,
    IN p_delivery_address text,
    IN p_phone text,
    IN p_email text,
    IN p_tax_id text
)
LANGUAGE plpgsql
AS $$
BEGIN
    IF p_company_name IS NULL OR trim(p_company_name) = '' THEN
        RAISE EXCEPTION 'Company name cannot be empty';
    END IF;

    IF p_email IS NOT NULL AND position('@' IN p_email) = 0 THEN
        RAISE EXCEPTION 'Invalid email format: %', p_email;
    END IF;

    INSERT INTO clients(
        company_name, delivery_address, phone, email, tax_id
    )
    VALUES (
        p_company_name, p_delivery_address, p_phone, p_email, p_tax_id
    );
END;
$$;
\end{lstlisting}

\subsection{Operacja Read}

Operacja \textit{Read} umożliwia odczyt danych klienta z bazy danych.
Została ona zaimplementowana w postaci funkcji składowanej,
która na podstawie identyfikatora klienta zwraca kompletny rekord
zawierający jego dane. W przypadku próby odczytu nieistniejącego rekordu
funkcja zgłasza wyjątek informujący o braku danych.

\begin{lstlisting}[language=SQL, caption={Funkcja odczytu danych klienta}]
CREATE OR REPLACE FUNCTION public.clients_read_one(
    p_client_id bigint
)
RETURNS public.clients
LANGUAGE plpgsql
AS $$
DECLARE
    v_row clients;
BEGIN
    SELECT *
    INTO v_row
    FROM clients
    WHERE client_id = p_client_id;

    IF NOT FOUND THEN
        RAISE EXCEPTION 'Client with id % not found', p_client_id;
    END IF;

    RETURN v_row;
END;
$$;
\end{lstlisting}

\subsection{Operacja Update}

Operacja \textit{Update} odpowiada za aktualizację istniejących danych
klienta. Została zrealizowana w formie procedury składowanej,
która umożliwia modyfikację wybranych pól rekordu.
Procedura sprawdza istnienie klienta w bazie danych oraz wykonuje
podstawową walidację danych wejściowych przed dokonaniem aktualizacji.

\begin{lstlisting}[language=SQL, caption={Procedura aktualizacji danych klienta}]
CREATE OR REPLACE PROCEDURE public.clients_update(
    IN p_client_id bigint,
    IN p_company_name text,
    IN p_delivery_address text,
    IN p_phone text,
    IN p_email text,
    IN p_tax_id text
)
LANGUAGE plpgsql
AS $$
BEGIN
    IF NOT EXISTS (
        SELECT 1 FROM clients WHERE client_id = p_client_id
    ) THEN
        RAISE EXCEPTION 'Client with id % does not exist', p_client_id;
    END IF;

    UPDATE clients
    SET company_name     = COALESCE(p_company_name, company_name),
        delivery_address = COALESCE(p_delivery_address, delivery_address),
        phone            = COALESCE(p_phone, phone),
        email            = COALESCE(p_email, email),
        tax_id           = COALESCE(p_tax_id, tax_id)
    WHERE client_id = p_client_id;
END;
$$;
\end{lstlisting}

\subsection{Operacja Delete}

Operacja \textit{Delete} służy do usuwania danych klienta z systemu.
Została ona zaimplementowana w postaci procedury składowanej,
która przed usunięciem rekordu sprawdza jego istnienie.
Dodatkowo obsługiwana jest sytuacja naruszenia integralności referencyjnej,
np. gdy klient jest powiązany z wysyłkami, co uniemożliwia jego usunięcie.

\begin{lstlisting}[language=SQL, caption={Procedura usuwania klienta}]
CREATE OR REPLACE PROCEDURE public.clients_delete(
    IN p_client_id bigint
)
LANGUAGE plpgsql
AS $$
BEGIN
    IF NOT EXISTS (
        SELECT 1 FROM clients WHERE client_id = p_client_id
    ) THEN
        RAISE EXCEPTION 'Client with id % does not exist', p_client_id;
    END IF;

    DELETE FROM clients WHERE client_id = p_client_id;

EXCEPTION
    WHEN foreign_key_violation THEN
        RAISE EXCEPTION
            'Cannot delete client %, client is used in shipments',
            p_client_id;
END;
$$;
\end{lstlisting}
