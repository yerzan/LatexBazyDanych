\documentclass{urdpl}     % praca w języku polskim

% Lista wszystkich języków stanowiących języki pozycji bibliograficznych użytych w pracy.
% (Zgodnie z zasadami tworzenia bibliografii każda pozycja powinna zostać utworzona zgodnie z zasadami języka, w którym dana publikacja została napisana.)
\usepackage[english,polish]{babel}

% Użyj polskiego łamania wyrazów (zamiast domyślnego angielskiego).
\usepackage{polski}

\usepackage[utf8]{inputenc}

% dodatkowe pakiety
\usepackage{pdfpages}
\usepackage{mathtools}
\usepackage{amsfonts}
\usepackage{amsmath}
\usepackage{amsthm}
\usepackage[hidelinks]{hyperref}
\usepackage{float}
\usepackage{listings}
\usepackage{graphicx}
\usepackage{subcaption}
\usepackage{booktabs} % Dla \toprule, \midrule, \bottomrule
\usepackage{multirow} 
\usepackage{tabularx} 
\usepackage{amssymb} 
\usepackage{listings}
\usepackage{xcolor}
\usepackage{array}
\usepackage{makecell}
\usepackage[flushleft]{threeparttable}
\usepackage[normalem]{ulem}
\usepackage{lineno}

% --- < bibliografia > ---
\usepackage{csquotes}

% --- < listingi > ---

% Użyj czcionki kroju Courier.
\usepackage{courier}

\usepackage{listings}
\lstloadlanguages{TeX}
\renewcommand{\lstlistlistingname}{Spis listingów}
\renewcommand{\lstlistingname}{Listing}

\lstset{
	literate={ą}{{\k{a}}}1
           {ć}{{\'c}}1
           {ę}{{\k{e}}}1
           {ó}{{\'o}}1
           {ń}{{\'n}}1
           {ł}{{\l{}}}1
           {ś}{{\'s}}1
           {ź}{{\'z}}1
           {ż}{{\..z}}1
           {\u0104}{{\k{A}}}1
           {\u0106}{{\'C}}1
           {\u0118}{{\k{E}}}1
           {\u00d3}{{\'O}}1
           {\u0143}{{\'N}}1
           {\u0141}{{\L{}}}1
           {\u015a}{{\'S}}1
           {\u0179}{{\'Z}}1
           {\u017b}{{\..Z}}1,
	basicstyle=\footnotesize\ttfamily,
}

% styl dla kodu JAVA
\lstdefinestyle{javaStyle}{
    language=Java,
    basicstyle=\ttfamily\footnotesize,
    keywordstyle=\color{blue},
    commentstyle=\color{green!50!black}\itshape,
    stringstyle=\color{green},
    numberstyle=\tiny\color{gray},
    numbers=left,
    numbersep=5pt,
    stepnumber=1,
    showspaces=false,
    tabsize=2,
    showstringspaces=false,
    breaklines=true,
    breakatwhitespace=false,
    showtabs=false,
    keepspaces=true
}

\definecolor{stringcolor}{RGB}{163,21,21}
\definecolor{typecolor}{RGB}{43, 145, 176}



% --- < podpisy > ---
\AtBeginDocument{
	\renewcommand{\tablename}{Tabela}
	\renewcommand{\figurename}{Rys.}   
    \newcommand{\listingname}{Listing}
}

% --- < tabele > ---
\newcolumntype{C}[1]{>{\hsize=#1\hsize\centering\arraybackslash}X}

% --- < dane strony tytułowej > ---
\author{Yevhen Marchak \& Mykhailo Kleban}
\shortauthor{Y. Marchak, M. Kleban}
\noAlbum{134945, 134922}

\titlePL{Dokumentacja projektu System zarządzania magazynem}
\titleEN{Dokumentacja projektu System zarządzania magazynem}

\shorttitlePL{Dokumentacja projektu SZM}
\shorttitleEN{Dokumentacja projektu SZM}

\thesistype{Praca projektowa}

\thesisDone{Praca wykonana pod kierunkiem}
\supervisor{dr inz. Piotr Grochowalski}

\degreeprogramme{Bazy danych}
\date{2026}

\department{Instytut Informatyki}
\faculty{Wydział Nauk Ściślych i Technicznych}

\setlength{\cftsecnumwidth}{10mm}
\setcounter{secnumdepth}{4}
\brokenpenalty=10000\relax

% --- < początek dokumentu > ---
\begin{document}

\titlepages

\fancypagestyle{plain}{
    \fancyhf{}
    \renewcommand{\headrulewidth}{0pt}
    \renewcommand{\footrulewidth}{0pt}
}

\setcounter{tocdepth}{2}
\tableofcontents
\clearpage

% --- < rozdziały > ---
\chapter{Specyfikacja tematu projektu}

\section{Cel projektu}

Celem projektu było zaprojektowanie oraz implementacja relacyjnej bazy danych
dla systemu zarządzania magazynem.
Projekt skupia się na stworzeniu spójnej, logicznej oraz znormalizowanej struktury
danych, umożliwiającej efektywne przechowywanie i przetwarzanie informacji
związanych z funkcjonowaniem magazynu.

Głównym założeniem projektu było opracowanie bazy danych, która umożliwia
zarządzanie stanami magazynowymi, produktami, kategoriami, klientami,
pracownikami oraz procesami przyjęcia i wydania towarów.
System wspiera realizację podstawowych operacji na danych (CRUD),
a także zapytań problemowych pozwalających na analizę danych magazynowych.

Celem jest praktyczne zastosowanie
wiedzy z zakresu projektowania relacyjnych baz danych, definiowania relacji
pomiędzy encjami, normalizacji danych oraz integracji bazy danych
z warstwą aplikacyjną.


\section{Zakres projektu}

Zakres projektu obejmuje zaprojektowanie oraz implementację relacyjnej bazy danych
systemu zarządzania magazynem.
Baza danych umożliwia przechowywanie informacji niezbędnych do obsługi
procesów magazynowych oraz zapewnia integralność i spójność danych.

W ramach projektu zrealizowano:
\begin{itemize}
    \item definicję struktur tabel wraz z kluczami głównymi i obcymi,
    \item zaprojektowanie relacji pomiędzy encjami systemu,
    \item implementację operacji CRUD dla kluczowych tabel,
    \item realizację zapytań problemowych o charakterze analitycznym,
    \item opracowanie diagramu związków encji (ERD),
    \item integrację bazy danych z aplikacją posiadającą interfejs graficzny.
\end{itemize}

Projekt obejmuje warstwę bazodanową oraz aplikacyjną.
Interfejs użytkownika umożliwia wygodną obsługę systemu,
natomiast logika biznesowa opiera się na zapytaniach SQL
wykonywanych na relacyjnej bazie danych.


\section{Technologie i narzędzia}

Do realizacji projektu wykorzystano następujące technologie oraz narzędzia:
\begin{itemize}
    \item \textbf{PostgreSQL} – relacyjny system zarządzania bazą danych,
    \item \textbf{pgAdmin4} – narzędzie administracyjne do zarządzania bazą danych,
    \item \textbf{SQL} – język zapytań do definiowania struktury bazy danych
    oraz operacji na danych,
    \item \textbf{JetBrains IntelliJ IDEA} – środowisko programistyczne
    wykorzystane do implementacji aplikacji, logiki systemu oraz interfejsu
    graficznego użytkownika (GUI),
    \item \textbf{Visual Studio Code} – środowisko używane do tworzenia
    dokumentacji projektu w systemie \LaTeX.
    \item \textbf{JDBC} – interfejs komunikacji aplikacji Java z bazą danych,
\end{itemize}


\section{Instrukcja uruchomienia projektu}

Aby poprawnie uruchomić system na lokalnym środowisku, należy postępować
zgodnie z poniższymi krokami:

\begin{enumerate}
    \item \textbf{Pobranie kodu źródłowego:}
    \begin{itemize}
        \item Pobierz projekt z repozytorium GitHub, klikając przycisk \textit{Code},
        a następnie wybierając opcję \textit{Download ZIP}.
        \item Rozpakuj pobrany folder ProjectBazyDanych-2rok-.
        \item Otwórz rozpakowany folder w wybranym środowisku programistycznym
        (zalecane: JetBrains IntelliJ IDEA).
    \end{itemize}

    \item \textbf{Przygotowanie relacyjnej bazy danych (PostgreSQL):}
    \begin{itemize}
        \item Uruchom narzędzie pgAdmin4 i utwórz nową, pustą bazę danych o nazwie SystemZarzadzaniaMagazynem.
        \item Kliknij prawym przyciskiem myszy na utworzoną bazę danych i wybierz
        opcję \textit{Restore}. Wybierz format Plain, i w pole Filename wklej plik "BD\_file.sql" i zatem nacisnij na przycisk \textit{Restore}.
        \item Kliknij prawym przyciskiem myszy na utworzoną bazę danych i wybierz
        opcję \textit{Refresh}.
    \end{itemize}

    \item \textbf{Konfiguracja połączenia z bazą danych:}
    \begin{itemize}
        \item Otwórz projekt aplikacji w środowisku JetBrains IntelliJ IDEA.
        \item Skonfiguruj parametry połączenia z bazą danych w pliku "DataBaseConnection", takie jak:
        nazwa bazy danych, użytkownik oraz hasło.
        \item Upewnij się, że aplikacja posiada poprawny dostęp do serwera
        PostgreSQL.
    \end{itemize}

    \item \textbf{Uruchomienie aplikacji:}
    \begin{itemize}
        \item Upewnij się, że serwer bazy danych PostgreSQL jest uruchomiony.
        \item Uruchom aplikację z poziomu środowiska JetBrains IntelliJ IDEA. Uruchomić plik Main.
        \item Po uruchomieniu aplikacji możliwa jest interakcja z systemem
        poprzez interfejs graficzny użytkownika (GUI).
    \end{itemize}
\end{enumerate}

Po wykonaniu powyższych kroków system zarządzania magazynem jest gotowy
do pracy i umożliwia realizację operacji na danych magazynowych.
 
\chapter{Aspekt projektowy baz danych}


\section{Zdefiniowane struktury tabel}

W niniejszym podrozdziale przedstawiono szczegółowy opis tabel
wchodzących w skład relacyjnej bazy danych systemu zarządzania magazynem.
Każda tabela została zaprojektowana w sposób zapewniający spójność danych
oraz poprawne odwzorowanie procesów magazynowych.

\subsection{Tabela \texttt{categories}}

Tabela \texttt{categories} przechowuje informacje o kategoriach produktów
dostępnych w magazynie.
\begin{itemize}
    \item \texttt{category\_id} (\texttt{BIGSERIAL}, PK) — unikalny identyfikator kategorii,
    \item \texttt{name} (\texttt{text}, NOT NULL) — nazwa kategorii,
    \item \texttt{description} (\texttt{text}) — opis kategorii.
\end{itemize}

\subsection{Tabela \texttt{suppliers}}

Tabela \texttt{suppliers} zawiera dane dotyczące dostawców towarów.
\begin{itemize}
    \item \texttt{supplier\_id} (\texttt{BIGSERIAL}, PK) — identyfikator dostawcy,
    \item \texttt{company\_name} (\texttt{text}, NOT NULL) — nazwa firmy dostawcy,
    \item \texttt{address} (\texttt{text}) — adres dostawcy,
    \item \texttt{phone} (\texttt{text}) — numer telefonu,
    \item \texttt{email} (\texttt{text}) — adres e-mail,
    \item \texttt{tax\_id} (\texttt{text}) — numer podatkowy.
\end{itemize}

\subsection{Tabela \texttt{clients}}

Tabela \texttt{clients} przechowuje dane klientów systemu magazynowego.
\begin{itemize}
    \item \texttt{client\_id} (\texttt{BIGSERIAL}, PK) — identyfikator klienta,
    \item \texttt{company\_name} (\texttt{text}, NOT NULL) — nazwa firmy klienta,
    \item \texttt{delivery\_address} (\texttt{text}) — adres dostawy,
    \item \texttt{phone} (\texttt{text}) — numer telefonu,
    \item \texttt{email} (\texttt{text}) — adres e-mail,
    \item \texttt{tax\_id} (\texttt{text}) — numer podatkowy.
\end{itemize}

\subsection{Tabela \texttt{employees}}

Tabela \texttt{employees} zawiera informacje o pracownikach magazynu.
\begin{itemize}
    \item \texttt{employee\_id} (\texttt{BIGSERIAL}, PK) — identyfikator pracownika,
    \item \texttt{first\_name} (\texttt{text}, NOT NULL) — imię,
    \item \texttt{last\_name} (\texttt{text}, NOT NULL) — nazwisko,
    \item \texttt{position} (\texttt{varchar(100)}, NOT NULL) — stanowisko,
    \item \texttt{hire\_date} (\texttt{date}) — data zatrudnienia,
    \item \texttt{phone} (\texttt{text}) — numer telefonu,
    \item \texttt{email} (\texttt{text}) — adres e-mail.
\end{itemize}

\subsection{Tabela \texttt{locations}}

Tabela \texttt{locations} definiuje lokalizacje magazynowe.
\begin{itemize}
    \item \texttt{location\_id} (\texttt{BIGSERIAL}, PK) — identyfikator lokalizacji,
    \item \texttt{location\_code} (\texttt{text}, NOT NULL) — kod lokalizacji,
    \item \texttt{location\_type} (\texttt{varchar(50)}, NOT NULL) — typ lokalizacji,
    \item \texttt{max\_capacity} (\texttt{numeric}) — maksymalna pojemność lokalizacji.
\end{itemize}

\subsection{Tabela \texttt{shipments}}

Tabela \texttt{shipments} rejestruje wydania towarów z magazynu do klientów.
Zawiera informacje dotyczące realizowanych wysyłek oraz osób odpowiedzialnych
za ich obsługę.

\begin{itemize}
    \item \texttt{shipment\_id} (\texttt{BIGSERIAL}, PK) — identyfikator wydania,
    \item \texttt{client\_id} (\texttt{bigint}, NOT NULL, FK $\rightarrow$ \texttt{clients.client\_id}) — klient,
    \item \texttt{employee\_id} (\texttt{bigint}, FK $\rightarrow$ \texttt{employees.employee\_id}) — pracownik realizujący wydanie,
    \item \texttt{shipment\_date} (\texttt{date}, NOT NULL) — data wydania,
    \item \texttt{status} (\texttt{varchar(50)}, NOT NULL) — status wydania
    (np. „oczekujące”, „zrealizowane”).
\end{itemize}

\subsection{Tabela \texttt{shipment\_details}}

Tabela \texttt{shipment\_details} przechowuje szczegółowe informacje dotyczące
produktów wydawanych w ramach konkretnego wydania magazynowego.
Stanowi tabelę pośrednią realizującą relację wiele do wielu pomiędzy
wydaniami a produktami.

\begin{itemize}
    \item \texttt{shipment\_id} (\texttt{bigint}, NOT NULL, FK $\rightarrow$ \texttt{shipments.shipment\_id}),
    \item \texttt{product\_id} (\texttt{bigint}, NOT NULL, FK $\rightarrow$ \texttt{products.product\_id}),
    \item \texttt{quantity} (\texttt{numeric}, NOT NULL) — ilość wydanego produktu.
\end{itemize}

\subsection{Tabela \texttt{products}}

Tabela \texttt{products} przechowuje informacje o produktach.
\begin{itemize}
    \item \texttt{product\_id} (\texttt{BIGSERIAL}, PK) — identyfikator produktu,
    \item \texttt{sku} (\texttt{text}, NOT NULL) — kod SKU produktu,
    \item \texttt{name} (\texttt{text}, NOT NULL) — nazwa produktu,
    \item \texttt{description} (\texttt{text}) — opis produktu,
    \item \texttt{category\_id} (\texttt{bigint}, FK $\rightarrow$ \texttt{categories.category\_id}) — kategoria produktu,
    \item \texttt{supplier\_id} (\texttt{bigint}, FK $\rightarrow$ \texttt{suppliers.supplier\_id}) — dostawca produktu,
    \item \texttt{weight} (\texttt{numeric}) — waga produktu,
    \item \texttt{dimensions} (\texttt{text}) — wymiary produktu.
\end{itemize}

\subsection{Tabela \texttt{inventory}}

Tabela \texttt{inventory} przechowuje informacje o stanach magazynowych.
\begin{itemize}
    \item \texttt{inventory\_id} (\texttt{BIGSERIAL}, PK) — identyfikator rekordu magazynowego,
    \item \texttt{product\_id} (\texttt{bigint}, FK $\rightarrow$ \texttt{products.product\_id}) — produkt,
    \item \texttt{location\_id} (\texttt{bigint}, FK $\rightarrow$ \texttt{locations.location\_id}) — lokalizacja magazynowa,
    \item \texttt{quantity} (\texttt{numeric}, NOT NULL) — ilość produktu,
    \item \texttt{last\_updated} (\texttt{timestamptz}) — data ostatniej aktualizacji.
\end{itemize}

\subsection{Tabela \texttt{receipts}}

Tabela \texttt{receipts} rejestruje przyjęcia towarów do magazynu.
\begin{itemize}
    \item \texttt{receipt\_id} (\texttt{BIGSERIAL}, PK) — identyfikator przyjęcia,
    \item \texttt{supplier\_id} (\texttt{bigint}, FK $\rightarrow$ \texttt{suppliers.supplier\_id}) — dostawca,
    \item \texttt{employee\_id} (\texttt{bigint}, FK $\rightarrow$ \texttt{employees.employee\_id}) — pracownik przyjmujący,
    \item \texttt{receipt\_date} (\texttt{date}, NOT NULL) — data przyjęcia,
    \item \texttt{external\_invoice\_no} (\texttt{text}) — numer faktury zewnętrznej,
    \item \texttt{status} (\texttt{varchar(50)}, NOT NULL) — status przyjęcia.
\end{itemize}

\subsection{Tabela \texttt{receipt\_details}}

Tabela \texttt{receipt\_details} zawiera szczegóły przyjęć magazynowych.
\begin{itemize}
    \item \texttt{receipt\_id} (\texttt{bigint}, FK $\rightarrow$ \texttt{receipts.receipt\_id}),
    \item \texttt{product\_id} (\texttt{bigint}, FK $\rightarrow$ \texttt{products.product\_id}),
    \item \texttt{expected\_quantity} (\texttt{numeric}, NOT NULL) — ilość oczekiwana,
    \item \texttt{received\_quantity} (\texttt{numeric}) — ilość faktycznie otrzymana.
\end{itemize}

\section{Diagram związków encji}

Struktura relacyjnej bazy danych systemu zarządzania magazynem
została przedstawiona w postaci diagramu związków encji (ERD).
Diagram obrazuje encje występujące w systemie oraz relacje
pomiędzy nimi, w tym klucze główne oraz obce.

Diagram ERD stanowi graficzne odwzorowanie struktury bazy danych
i ułatwia analizę zależności pomiędzy poszczególnymi tabelami.

\begin{figure}[H]
    \centering
    \includegraphics[width=0.95\textwidth]{figures/ERD.jpg}
    \caption{Diagram związków encji (ERD) systemu zarządzania magazynem}
    \label{fig:erd}
\end{figure}

\section{Powiązania pomiędzy tabelami}

Relacje pomiędzy tabelami w systemie zarządzania magazynem zostały
zaprojektowane w sposób odzwierciedlający rzeczywiste zależności
występujące w procesach magazynowych.
Zastosowanie kluczy obcych zapewnia integralność referencyjną danych
oraz eliminuje możliwość występowania niespójnych rekordów.

Poniżej przedstawiono najważniejsze relacje pomiędzy encjami systemu:

\begin{itemize}
    \item Relacja typu jeden do wielu (1:N) pomiędzy tabelą
    \texttt{categories} a tabelą \texttt{products}.
    Jedna kategoria może być przypisana do wielu produktów,
    natomiast każdy produkt należy do jednej kategorii.

    \item Relacja typu jeden do wielu (1:N) pomiędzy tabelą
    \texttt{suppliers} a tabelą \texttt{products}.
    Jeden dostawca może dostarczać wiele produktów,
    natomiast każdy produkt posiada jednego dostawcę.

    \item Relacja typu jeden do wielu (1:N) pomiędzy tabelą
    \texttt{locations} a tabelą \texttt{inventory}.
    Jedna lokalizacja magazynowa może zawierać wiele rekordów stanów magazynowych.

    \item Relacja typu jeden do wielu (1:N) pomiędzy tabelą
    \texttt{products} a tabelą \texttt{inventory}.
    Każdy produkt może występować w wielu lokalizacjach magazynowych.

    \item Relacja typu jeden do wielu (1:N) pomiędzy tabelą
    \texttt{suppliers} a tabelą \texttt{receipts}.
    Jeden dostawca może realizować wiele przyjęć magazynowych.

    \item Relacja typu jeden do wielu (1:N) pomiędzy tabelą
    \texttt{employees} a tabelą \texttt{receipts}.
    Jeden pracownik może obsługiwać wiele przyjęć towarów.

    \item Relacja typu wiele do wielu (N:M) pomiędzy tabelą
    \texttt{receipts} a tabelą \texttt{products},
    zrealizowana za pomocą tabeli pośredniej \texttt{receipt\_details}.
    Tabela ta przechowuje informacje o ilościach oczekiwanych oraz faktycznie
    otrzymanych produktów.

    \item Relacja typu jeden do wielu (1:N) pomiędzy tabelą
    \texttt{clients} a tabelą \texttt{shipments}.
    Jeden klient może posiadać wiele wydań magazynowych.

    \item Relacja typu jeden do wielu (1:N) pomiędzy tabelą
    \texttt{employees} a tabelą \texttt{shipments}.
    Jeden pracownik może realizować wiele wydań towarów.

    \item Relacja typu wiele do wielu (N:M) pomiędzy tabelą
    \texttt{shipments} a tabelą \texttt{products},
    zrealizowana za pomocą tabeli pośredniej \texttt{shipment\_details}.
    Tabela ta przechowuje informacje o ilości wydanych produktów.
\end{itemize}

Zaprojektowane relacje umożliwiają poprawne odwzorowanie procesów
przyjęcia oraz wydania towarów, a także zapewniają spójność
i integralność danych w całym systemie.

\section{Kod realizujący bazę danych}

W niniejszym podrozdziale przedstawiono kompletną definicję struktur
tabel relacyjnej bazy danych systemu zarządzania magazynem.
Zaprezentowane listingi zawierają instrukcje \texttt{CREATE TABLE}
odpowiedzialne za utworzenie wszystkich encji systemu wraz
z kluczami głównymi, kluczami obcymi oraz ograniczeniami integralności.

\begin{lstlisting}[language=SQL, caption={Definicja tabeli categories}]
CREATE TABLE categories (
    category_id BIGSERIAL PRIMARY KEY,
    name TEXT NOT NULL UNIQUE,
    description TEXT
);
\end{lstlisting}

\begin{lstlisting}[language=SQL, caption={Definicja tabeli suppliers}]
CREATE TABLE suppliers (
    supplier_id BIGSERIAL PRIMARY KEY,
    company_name TEXT NOT NULL,
    address TEXT,
    phone TEXT,
    email TEXT,
    tax_id TEXT
);
\end{lstlisting}

\begin{lstlisting}[language=SQL, caption={Definicja tabeli clients}]
CREATE TABLE clients (
    client_id BIGSERIAL PRIMARY KEY,
    company_name TEXT NOT NULL,
    delivery_address TEXT,
    phone TEXT,
    email TEXT,
    tax_id TEXT
);
\end{lstlisting}

\begin{lstlisting}[language=SQL, caption={Definicja tabeli employees}]
CREATE TABLE employees (
    employee_id BIGSERIAL PRIMARY KEY,
    first_name TEXT NOT NULL,
    last_name TEXT NOT NULL,
    position VARCHAR(100) NOT NULL,
    hire_date DATE,
    phone TEXT,
    email TEXT
);
\end{lstlisting}

\begin{lstlisting}[language=SQL, caption={Definicja tabeli locations}]
CREATE TABLE locations (
    location_id BIGSERIAL PRIMARY KEY,
    location_code TEXT NOT NULL UNIQUE,
    location_type VARCHAR(50) NOT NULL,
    max_capacity NUMERIC CHECK (max_capacity >= 0)
);
\end{lstlisting}

\begin{lstlisting}[language=SQL, caption={Definicja tabeli products}]
CREATE TABLE products (
    product_id BIGSERIAL PRIMARY KEY,
    sku TEXT NOT NULL UNIQUE,
    name TEXT NOT NULL,
    description TEXT,
    category_id BIGINT NOT NULL,
    supplier_id BIGINT,
    weight NUMERIC CHECK (weight >= 0),
    dimensions TEXT,
    FOREIGN KEY (category_id) REFERENCES categories(category_id),
    FOREIGN KEY (supplier_id) REFERENCES suppliers(supplier_id)
);
\end{lstlisting}

\begin{lstlisting}[language=SQL, caption={Definicja tabeli inventory}]
CREATE TABLE inventory (
    inventory_id BIGSERIAL PRIMARY KEY,
    product_id BIGINT NOT NULL,
    location_id BIGINT NOT NULL,
    quantity NUMERIC NOT NULL CHECK (quantity >= 0),
    last_updated TIMESTAMPTZ DEFAULT now(),
    UNIQUE (product_id, location_id),
    FOREIGN KEY (product_id) REFERENCES products(product_id),
    FOREIGN KEY (location_id) REFERENCES locations(location_id)
);
\end{lstlisting}

\begin{lstlisting}[language=SQL, caption={Definicja tabeli receipts}]
CREATE TABLE receipts (
    receipt_id BIGSERIAL PRIMARY KEY,
    supplier_id BIGINT NOT NULL,
    employee_id BIGINT,
    receipt_date DATE DEFAULT CURRENT_DATE,
    external_invoice_no TEXT,
    status VARCHAR(50) NOT NULL,
    FOREIGN KEY (supplier_id) REFERENCES suppliers(supplier_id),
    FOREIGN KEY (employee_id) REFERENCES employees(employee_id)
);
\end{lstlisting}

\begin{lstlisting}[language=SQL, caption={Definicja tabeli receipt\_details}]
CREATE TABLE receipt_details (
    receipt_id BIGINT NOT NULL,
    product_id BIGINT NOT NULL,
    expected_quantity NUMERIC NOT NULL CHECK (expected_quantity >= 0),
    received_quantity NUMERIC NOT NULL CHECK (received_quantity >= 0),
    PRIMARY KEY (receipt_id, product_id),
    FOREIGN KEY (receipt_id) REFERENCES receipts(receipt_id),
    FOREIGN KEY (product_id) REFERENCES products(product_id)
);
\end{lstlisting}

\begin{lstlisting}[language=SQL, caption={Definicja tabeli shipments}]
CREATE TABLE shipments (
    shipment_id BIGSERIAL PRIMARY KEY,
    client_id BIGINT NOT NULL,
    employee_id BIGINT,
    shipment_date DATE DEFAULT CURRENT_DATE,
    client_order_no TEXT,
    status VARCHAR(50) NOT NULL,
    FOREIGN KEY (client_id) REFERENCES clients(client_id),
    FOREIGN KEY (employee_id) REFERENCES employees(employee_id)
);
\end{lstlisting}

\begin{lstlisting}[language=SQL, caption={Definicja tabeli shipment\_details}]
CREATE TABLE shipment_details (
    shipment_id BIGINT NOT NULL,
    product_id BIGINT NOT NULL,
    quantity_to_ship NUMERIC NOT NULL CHECK (quantity_to_ship >= 0),
    PRIMARY KEY (shipment_id, product_id),
    FOREIGN KEY (shipment_id) REFERENCES shipments(shipment_id),
    FOREIGN KEY (product_id) REFERENCES products(product_id)
);
\end{lstlisting}

\section{Normalizacja bazy danych}

Proces normalizacji bazy danych został przeprowadzony w celu eliminacji redundancji danych,
zapewnienia spójności informacji oraz poprawy integralności logicznej systemu.
Projektowana baza danych systemu zarządzania magazynem spełnia założenia trzech
pierwszych postaci normalnych: 1NF, 2NF oraz 3NF.

\subsection{Pierwsza postać normalna (1NF)}

Pierwsza postać normalna wymaga, aby:
\begin{itemize}
    \item wszystkie atrybuty zawierały wartości atomowe,
    \item w tabelach nie występowały powtarzające się grupy danych,
    \item każdy rekord był jednoznacznie identyfikowany przez klucz główny.
\end{itemize}

W zaprojektowanej bazie danych wszystkie tabele posiadają klucze główne
(np. \texttt{category\_id}, \texttt{product\_id}, \texttt{shipment\_id}),
a każdy atrybut przechowuje pojedynczą, niepodzielną wartość.
Nie występują pola wielowartościowe ani listy danych w jednym atrybucie.

\subsection{Druga postać normalna (2NF)}

Druga postać normalna wymaga spełnienia warunków 1NF oraz braku zależności
częściowych atrybutów niekluczowych od części klucza głównego.

W bazie danych zależności częściowe zostały wyeliminowane poprzez:
\begin{itemize}
    \item rozdzielenie danych opisowych do osobnych tabel (np. \texttt{categories}, \texttt{suppliers}),
    \item zastosowanie tabel pośrednich z kluczami złożonymi, takich jak
    \texttt{receipt\_details} oraz \texttt{shipment\_details}.
\end{itemize}

W tabelach posiadających klucz złożony wszystkie atrybuty niekluczowe
zależą od całego klucza głównego, a nie od jego fragmentu.

\subsection{Trzecia postać normalna (3NF)}

Trzecia postać normalna wymaga, aby w tabelach nie występowały zależności
przechodnie pomiędzy atrybutami niekluczowymi.

W zaprojektowanej bazie danych:
\begin{itemize}
    \item dane klientów, pracowników, dostawców i produktów zostały umieszczone
    w oddzielnych tabelach,
    \item informacje zależne od innych encji są przechowywane wyłącznie poprzez
    klucze obce,
    \item nie występują atrybuty, które mogłyby być wyznaczane na podstawie innych
    atrybutów niekluczowych w tej samej tabeli.
\end{itemize}

Dzięki temu baza danych spełnia wymagania trzeciej postaci normalnej,
zapewniając wysoką spójność danych oraz łatwość dalszej rozbudowy systemu.

\subsection{Podsumowanie}

Zaprojektowana struktura bazy danych systemu zarządzania magazynem
została poprawnie znormalizowana do trzeciej postaci normalnej (3NF).
Zastosowana normalizacja minimalizuje redundancję danych, ułatwia
utrzymanie spójności informacji oraz poprawia wydajność operacji
na bazie danych.
 
\chapter{Zapytania problemowe i ich realizacja}

W niniejszym rozdziale przedstawiono zapytania problemowe,
które odzwierciedlają rzeczywiste potrzeby biznesowe systemu
zarządzania magazynem. Dla każdego problemu zaprezentowano
opis zagadnienia, kolejne kroki rozwiązania oraz implementację
w postaci funkcji lub procedury składowanej w języku PL/pgSQL.

\section{Premia dla pracowników za obsłużone wysyłki}
\begin{itemize}
    \item wyszukanie pracowników realizujących wysyłki w danym miesiącu,
    \item zliczenie liczby wysyłek dla każdego pracownika,
    \item wyznaczenie procentu premii na podstawie liczby wysyłek.
\end{itemize}

\begin{lstlisting}[language=SQL, caption={Funkcja obliczająca premię pracownika}]
CREATE OR REPLACE FUNCTION public.calculate_employee_bonus(
    p_month INT,
    p_year INT
)
RETURNS TABLE (
    employee_id BIGINT,
    shipments_count INT,
    bonus_percent INT
)
LANGUAGE plpgsql
AS $$
BEGIN
    RETURN QUERY
    SELECT
        s.employee_id,
        COUNT(*)::INT AS shipments_count,
        CASE
            WHEN COUNT(*) > 100 THEN 30
            WHEN COUNT(*) BETWEEN 50 AND 100 THEN 15
            ELSE 0
        END AS bonus_percent
    FROM public.shipments s
    WHERE EXTRACT(MONTH FROM s.shipment_date) = p_month
      AND EXTRACT(YEAR  FROM s.shipment_date) = p_year
      AND s.employee_id IS NOT NULL
    GROUP BY s.employee_id;
END;
$$;
\end{lstlisting}

\section{Automatyczne oznaczanie przeterminowanych wysyłek}
\begin{itemize}
    \item wyszukanie wysyłek starszych niż 7 dni,
    \item sprawdzenie ich aktualnego statusu,
    \item aktualizacja statusu na \texttt{OVERDUE}.
\end{itemize}

\begin{lstlisting}[language=SQL, caption={Procedura oznaczania przeterminowanych wysyłek}]
CREATE OR REPLACE PROCEDURE public.mark_overdue_shipments()
LANGUAGE plpgsql
AS $$
BEGIN
    UPDATE public.shipments
    SET status = 'OVERDUE'
    WHERE shipment_date < CURRENT_DATE - INTERVAL '7 days'
      AND status <> 'DELIVERED';
END;
$$;
\end{lstlisting}

\section{Wykrywanie braków magazynowych}
\begin{itemize}
    \item pobranie aktualnych stanów magazynowych,
    \item porównanie ilości z progiem minimalnym,
    \item zwrócenie produktów wymagających uzupełnienia.
\end{itemize}

\begin{lstlisting}[language=SQL, caption={Funkcja wykrywająca braki magazynowe}]
CREATE OR REPLACE FUNCTION public.find_low_stock_products(
    p_min_quantity NUMERIC
)
RETURNS TABLE (
    product_id BIGINT,
    location_id BIGINT,
    quantity NUMERIC
)
LANGUAGE plpgsql
AS $$
BEGIN
    RETURN QUERY
    SELECT
        i.product_id,
        i.location_id,
        i.quantity
    FROM public.inventory i
    WHERE i.quantity < p_min_quantity;
END;
$$;
\end{lstlisting}

\section{Obliczenie stopnia realizacji dostawy}
\begin{itemize}
    \item pobranie ilości oczekiwanych i faktycznie odebranych produktów,
    \item zsumowanie wartości dla całej dostawy,
    \item obliczenie procentowego stopnia realizacji przyjęcia.
\end{itemize}

\begin{lstlisting}[language=SQL, caption={Funkcja obliczająca stopień realizacji dostawy}]
CREATE OR REPLACE FUNCTION public.receipt_completion_percentage(
    p_receipt_id BIGINT
)
RETURNS NUMERIC
LANGUAGE plpgsql
AS $$
DECLARE
    expected_sum NUMERIC := 0;
    received_sum NUMERIC := 0;
BEGIN
    SELECT
        COALESCE(SUM(expected_quantity), 0),
        COALESCE(SUM(received_quantity), 0)
    INTO expected_sum, received_sum
    FROM public.receipt_details
    WHERE receipt_id = p_receipt_id;

    IF expected_sum = 0 THEN
        RETURN 0;
    END IF;

    RETURN ROUND((received_sum / expected_sum) * 100, 2);
END;
$$;
\end{lstlisting}

\section{Najaktywniejsi klienci w danym okresie}
\begin{itemize}
    \item pobranie wysyłek z wybranego okresu,
    \item zliczenie liczby wysyłek dla każdego klienta,
    \item wybranie klientów przekraczających ustalony próg aktywności.
\end{itemize}

\begin{lstlisting}[language=SQL, caption={Funkcja identyfikująca najaktywniejszych klientów}]
CREATE OR REPLACE FUNCTION public.get_top_clients(
    p_from DATE,
    p_to DATE,
    p_min_shipments INT
)
RETURNS TABLE (
    client_id BIGINT,
    shipments_count INT
)
LANGUAGE plpgsql
AS $$
BEGIN
    RETURN QUERY
    SELECT
        s.client_id,
        COUNT(*)::INT AS shipments_count
    FROM public.shipments s
    WHERE s.shipment_date BETWEEN p_from AND p_to
    GROUP BY s.client_id
    HAVING COUNT(*) >= p_min_shipments;
END;
$$;
\end{lstlisting}

\section{Kontrola pojemności lokalizacji magazynowej}
\begin{itemize}
    \item zsumowanie ilości produktów przypisanych do lokalizacji,
    \item pobranie maksymalnej pojemności lokalizacji,
    \item porównanie wartości i zwrócenie wyniku kontroli.
\end{itemize}

\begin{lstlisting}[language=SQL, caption={Funkcja kontrolująca pojemność lokalizacji}]
CREATE OR REPLACE FUNCTION public.check_location_capacity(
    p_location_id BIGINT
)
RETURNS BOOLEAN
LANGUAGE plpgsql
AS $$
DECLARE
    total_quantity NUMERIC := 0;
    v_max_capacity NUMERIC;
BEGIN
    SELECT COALESCE(SUM(i.quantity), 0)
    INTO total_quantity
    FROM public.inventory i
    WHERE i.location_id = p_location_id;

    SELECT l.max_capacity
    INTO v_max_capacity
    FROM public.locations l
    WHERE l.location_id = p_location_id;

    IF v_max_capacity IS NULL THEN
        RETURN TRUE;
    END IF;

    RETURN total_quantity <= v_max_capacity;
END;
$$;
\end{lstlisting}

\section{Aktualizacja stanu magazynu po przyjęciu dostawy}
\begin{itemize}
    \item pobranie danych o faktycznie odebranych ilościach produktów,
    \item dopasowanie produktów do odpowiednich rekordów magazynowych,
    \item aktualizacja ilości oraz daty ostatniej modyfikacji.
\end{itemize}

\begin{lstlisting}[language=SQL, caption={Procedura aktualizująca stany magazynowe po przyjęciu}]
CREATE OR REPLACE PROCEDURE public.update_inventory_after_receipt(
    p_receipt_id BIGINT
)
LANGUAGE plpgsql
AS $$
BEGIN
    UPDATE public.inventory i
    SET quantity = i.quantity + rd.received_quantity,
        last_updated = NOW()
    FROM public.receipt_details rd
    WHERE rd.receipt_id = p_receipt_id
      AND rd.product_id = i.product_id;
END;
$$;
\end{lstlisting}

\section{Ocena produktywności pracowników}
\begin{itemize}
    \item pobranie danych o wysyłkach realizowanych przez pracowników,
    \item zliczenie liczby unikalnych klientów,
    \item określenie poziomu produktywności pracownika.
\end{itemize}

\begin{lstlisting}[language=SQL, caption={Funkcja oceniająca produktywność pracowników}]
CREATE OR REPLACE FUNCTION public.employee_productivity()
RETURNS TABLE (
    employee_id BIGINT,
    clients_count INT,
    productivity_level TEXT
)
LANGUAGE plpgsql
AS $$
BEGIN
    RETURN QUERY
    SELECT
        s.employee_id,
        COUNT(DISTINCT s.client_id)::INT AS clients_count,
        CASE
            WHEN COUNT(DISTINCT s.client_id) > 20 THEN 'HIGH'
            WHEN COUNT(DISTINCT s.client_id) BETWEEN 10 AND 20 THEN 'MEDIUM'
            ELSE 'LOW'
        END AS productivity_level
    FROM public.shipments s
    WHERE s.employee_id IS NOT NULL
    GROUP BY s.employee_id;
END;
$$;
\end{lstlisting}

\section{Najczęściej wysyłane produkty}
\begin{itemize}
    \item pobranie danych dotyczących wysyłanych produktów,
    \item zsumowanie ilości wysyłek dla poszczególnych produktów,
    \item wybranie produktów przekraczających ustalony próg.
\end{itemize}

\begin{lstlisting}[language=SQL, caption={Funkcja identyfikująca najczęściej wysyłane produkty}]
CREATE OR REPLACE FUNCTION public.most_shipped_products(
    p_min_quantity NUMERIC
)
RETURNS TABLE (
    product_id BIGINT,
    total_quantity NUMERIC
)
LANGUAGE plpgsql
AS $$
BEGIN
    RETURN QUERY
    SELECT
        sd.product_id,
        SUM(sd.quantity_to_ship) AS total_quantity
    FROM public.shipment_details sd
    GROUP BY sd.product_id
    HAVING SUM(sd.quantity_to_ship) >= p_min_quantity;
END;
$$;
\end{lstlisting}

\section{Walidacja poprawności dostawy}
\begin{itemize}
    \item pobranie danych dotyczących ilości oczekiwanych i odebranych,
    \item sprawdzenie poprawności wartości,
    \item zatwierdzenie lub odrzucenie dostawy.
\end{itemize}

\begin{lstlisting}[language=SQL, caption={Funkcja walidująca poprawność dostawy}]
CREATE OR REPLACE FUNCTION public.validate_receipt(
    p_receipt_id BIGINT
)
RETURNS BOOLEAN
LANGUAGE plpgsql
AS $$
BEGIN
    IF EXISTS (
        SELECT 1
        FROM public.receipt_details
        WHERE receipt_id = p_receipt_id
          AND received_quantity > expected_quantity
    ) THEN
        RETURN FALSE;
    END IF;

    RETURN TRUE;
END;
$$;
\end{lstlisting}

\section{Wymagania CRUD systemu}

System zarządzania magazynem musi zapewniać pełną obsługę operacji
typu CRUD (Create, Read, Update, Delete) dla wszystkich kluczowych
encji występujących w bazie danych. Operacje te są niezbędne do
prawidłowego funkcjonowania systemu oraz realizacji zapytań
problemowych przedstawionych w dalszej części rozdziału.

\subsection*{Create (tworzenie danych)}
Operacje tworzenia danych umożliwiają dodawanie nowych rekordów
do bazy danych, takich jak:
\begin{itemize}
    \item nowi klienci i dostawcy,
    \item produkty oraz ich kategorie,
    \item lokalizacje magazynowe,
    \item przyjęcia towarów (receipts) oraz wysyłki (shipments),
    \item szczegóły przyjęć i wysyłek.
\end{itemize}
Operacje \texttt{CREATE} realizowane są przy użyciu procedur
składowanych, które dodatkowo wykonują walidację danych wejściowych.

\subsection*{Read (odczyt danych)}
Operacje odczytu danych umożliwiają pobieranie informacji
niezbędnych do analizy stanu magazynu i podejmowania decyzji
biznesowych. Obejmują one m.in.:
\begin{itemize}
    \item przegląd aktualnych stanów magazynowych,
    \item listę klientów, dostawców i pracowników,
    \item historię wysyłek i przyjęć towarów,
    \item dane wykorzystywane w zapytaniach analitycznych.
\end{itemize}
Operacje \texttt{READ} realizowane są za pomocą funkcji zwracających
pojedyncze rekordy lub zbiory danych.

\subsection*{Update (aktualizacja danych)}
Operacje aktualizacji danych umożliwiają modyfikację istniejących
rekordów w bazie danych, w szczególności:
\begin{itemize}
    \item zmianę statusów wysyłek i dostaw,
    \item aktualizację stanów magazynowych,
    \item korektę danych klientów, produktów i lokalizacji,
    \item rejestrację faktycznie odebranych ilości towarów.
\end{itemize}
Operacje \texttt{UPDATE} są realizowane przez procedury składowane,
co zapewnia spójność danych i kontrolę poprawności modyfikacji.

\subsection*{Delete (usuwanie danych)}
Operacje usuwania danych pozwalają na eliminację nieaktualnych
lub błędnie wprowadzonych rekordów, takich jak:
\begin{itemize}
    \item nieużywane produkty lub kategorie,
    \item błędnie dodane rekordy magazynowe,
    \item anulowane wysyłki lub przyjęcia.
\end{itemize}
Operacje \texttt{DELETE} są zabezpieczone mechanizmami integralności
referencyjnej, co zapobiega usuwaniu danych powiązanych z innymi
encjami systemu.

\medskip
Spełnienie powyższych wymagań CRUD stanowi podstawę do realizacji
zaawansowanych funkcjonalności systemu, w tym zapytań problemowych
opisanych w kolejnych podrozdziałach.
 
\chapter{Przedstawienie powstałej bazy danych i wdrożonych mechanizmów}

\section{Realizacja mechanizmu CRUD}

W systemie zarządzania magazynem zaimplementowano mechanizm CRUD
(\textit{Create, Read, Update, Delete}), który umożliwia wykonywanie
podstawowych operacji na danych przechowywanych w relacyjnej bazie danych
PostgreSQL. Mechanizm ten został zrealizowany przy użyciu procedur oraz
funkcji składowanych napisanych w języku \textit{PL/pgSQL}.

W celu czytelnego zaprezentowania działania mechanizmu CRUD,
jego implementację przedstawiono na przykładzie tabeli
\textbf{clients}, przechowującej dane klientów obsługiwanych przez system.
Dla tej tabeli przygotowano komplet procedur i funkcji umożliwiających
dodawanie, odczyt, aktualizację oraz usuwanie danych.

\subsection{Operacja Create}

Operacja \textit{Create} odpowiada za dodawanie nowych rekordów do tabeli
\texttt{clients}. Zrealizowano ją w postaci procedury składowanej,
która przyjmuje dane klienta jako parametry wejściowe.
Przed zapisaniem danych wykonywana jest walidacja poprawności
najważniejszych pól, takich jak nazwa firmy oraz adres e-mail.
Dzięki temu system zapobiega zapisywaniu niepoprawnych danych w bazie.

\begin{lstlisting}[language=SQL, caption={Procedura dodawania klienta}]
CREATE OR REPLACE PROCEDURE public.clients_create(
    IN p_company_name text,
    IN p_delivery_address text,
    IN p_phone text,
    IN p_email text,
    IN p_tax_id text
)
LANGUAGE plpgsql
AS $$
BEGIN
    IF p_company_name IS NULL OR trim(p_company_name) = '' THEN
        RAISE EXCEPTION 'Company name cannot be empty';
    END IF;

    IF p_email IS NOT NULL AND position('@' IN p_email) = 0 THEN
        RAISE EXCEPTION 'Invalid email format: %', p_email;
    END IF;

    INSERT INTO clients(
        company_name, delivery_address, phone, email, tax_id
    )
    VALUES (
        p_company_name, p_delivery_address, p_phone, p_email, p_tax_id
    );
END;
$$;
\end{lstlisting}

\subsection{Operacja Read}

Operacja \textit{Read} umożliwia odczyt danych klienta z bazy danych.
Została ona zaimplementowana w postaci funkcji składowanej,
która na podstawie identyfikatora klienta zwraca kompletny rekord
zawierający jego dane. W przypadku próby odczytu nieistniejącego rekordu
funkcja zgłasza wyjątek informujący o braku danych.

\begin{lstlisting}[language=SQL, caption={Funkcja odczytu danych klienta}]
CREATE OR REPLACE FUNCTION public.clients_read_one(
    p_client_id bigint
)
RETURNS public.clients
LANGUAGE plpgsql
AS $$
DECLARE
    v_row clients;
BEGIN
    SELECT *
    INTO v_row
    FROM clients
    WHERE client_id = p_client_id;

    IF NOT FOUND THEN
        RAISE EXCEPTION 'Client with id % not found', p_client_id;
    END IF;

    RETURN v_row;
END;
$$;
\end{lstlisting}

\subsection{Operacja Update}

Operacja \textit{Update} odpowiada za aktualizację istniejących danych
klienta. Została zrealizowana w formie procedury składowanej,
która umożliwia modyfikację wybranych pól rekordu.
Procedura sprawdza istnienie klienta w bazie danych oraz wykonuje
podstawową walidację danych wejściowych przed dokonaniem aktualizacji.

\begin{lstlisting}[language=SQL, caption={Procedura aktualizacji danych klienta}]
CREATE OR REPLACE PROCEDURE public.clients_update(
    IN p_client_id bigint,
    IN p_company_name text,
    IN p_delivery_address text,
    IN p_phone text,
    IN p_email text,
    IN p_tax_id text
)
LANGUAGE plpgsql
AS $$
BEGIN
    IF NOT EXISTS (
        SELECT 1 FROM clients WHERE client_id = p_client_id
    ) THEN
        RAISE EXCEPTION 'Client with id % does not exist', p_client_id;
    END IF;

    UPDATE clients
    SET company_name     = COALESCE(p_company_name, company_name),
        delivery_address = COALESCE(p_delivery_address, delivery_address),
        phone            = COALESCE(p_phone, phone),
        email            = COALESCE(p_email, email),
        tax_id           = COALESCE(p_tax_id, tax_id)
    WHERE client_id = p_client_id;
END;
$$;
\end{lstlisting}

\subsection{Operacja Delete}

Operacja \textit{Delete} służy do usuwania danych klienta z systemu.
Została ona zaimplementowana w postaci procedury składowanej,
która przed usunięciem rekordu sprawdza jego istnienie.
Dodatkowo obsługiwana jest sytuacja naruszenia integralności referencyjnej,
np. gdy klient jest powiązany z wysyłkami, co uniemożliwia jego usunięcie.

\begin{lstlisting}[language=SQL, caption={Procedura usuwania klienta}]
CREATE OR REPLACE PROCEDURE public.clients_delete(
    IN p_client_id bigint
)
LANGUAGE plpgsql
AS $$
BEGIN
    IF NOT EXISTS (
        SELECT 1 FROM clients WHERE client_id = p_client_id
    ) THEN
        RAISE EXCEPTION 'Client with id % does not exist', p_client_id;
    END IF;

    DELETE FROM clients WHERE client_id = p_client_id;

EXCEPTION
    WHEN foreign_key_violation THEN
        RAISE EXCEPTION
            'Cannot delete client %, client is used in shipments',
            p_client_id;
END;
$$;
\end{lstlisting}
 
\chapter{Koncepcja dostępu zdalnego do bazy danych}

\section{Założenia architektoniczne}

System Zarządzania Magazynem został zrealizowany jako aplikacja
desktopowa działająca w architekturze klient--serwer.
Aplikacja kliencka została napisana w języku Java i komunikuje się
bezpośrednio z serwerem bazy danych PostgreSQL.

Dostęp do bazy danych realizowany jest z wykorzystaniem mechanizmu
JDBC (Java Database Connectivity), który umożliwia wykonywanie
zapytań SQL oraz wywoływanie procedur i funkcji składowanych.

\section{Realizacja połączenia z bazą danych}

Połączenie aplikacji z bazą danych realizowane jest poprzez klasę
odpowiedzialną za konfigurację połączenia JDBC. Wykorzystywany jest
adres serwera, numer portu, nazwa bazy danych oraz dane użytkownika.

\section{Zakres dostępu do aplikacji}

Aplikacja została zaprojektowana jako narzędzie administracyjne
i przeznaczona jest wyłącznie do użytku przez administratora systemu.
Nie przewidziano mechanizmu obsługi wielu ról użytkowników ani
logowania końcowych użytkowników.

Administrator posiada pełny dostęp do wszystkich funkcjonalności
systemu, w tym do zarządzania danymi magazynowymi, obsługi procesów
przyjęć i wysyłek oraz wykonywania zapytań analitycznych.
Takie podejście upraszcza architekturę aplikacji oraz umożliwia
bezpośrednią kontrolę nad danymi przechowywanymi w bazie.

 
\chapter{Interfejs użytkownika aplikacji}

W niniejszym rozdziale przedstawiono interfejs użytkownika aplikacji
zarządzania magazynem. Zaprezentowane ekrany ilustrują najważniejsze
funkcjonalności systemu, w tym obsługę danych słownikowych, realizację
operacji CRUD oraz wizualizację zapytań problemowych.

\section{Główne okno aplikacji}

Główne okno aplikacji stanowi centralny punkt dostępu do wszystkich
funkcjonalności systemu. Użytkownik ma możliwość przechodzenia do
poszczególnych modułów, takich jak zarządzanie kategoriami, klientami,
pracownikami, produktami, lokalizacjami magazynowymi oraz realizacja
zapytań problemowych.

\begin{figure}[H]
    \centering
    \includegraphics[width=\textwidth]{figures/mainwindow.png}
    \caption{Główne okno aplikacji}
    \label{fig:mainwindow}
\end{figure}

\section{Zarządzanie kategoriami produktów}

Okno zarządzania kategoriami umożliwia wykonywanie operacji CRUD
(Create, Read, Update, Delete) na tabeli kategorii. Użytkownik może
przeglądać listę istniejących kategorii, dodawać nowe rekordy,
modyfikować istniejące dane oraz usuwać wybrane pozycje.

\begin{figure}[H]
    \centering
    \includegraphics[width=0.8\textwidth]{figures/categories.jpg}
    \caption{Okno zarządzania kategoriami}
    \label{fig:categories}
\end{figure}

\section{Zarządzanie klientami}

Moduł klientów umożliwia kompleksową obsługę danych kontrahentów.
Dostępne są operacje dodawania, edycji oraz usuwania klientów, a także
przegląd pełnej listy zapisanej w bazie danych. Dane prezentowane są
w formie tabelarycznej, co ułatwia ich analizę i wyszukiwanie.

\begin{figure}[H]
    \centering
    \includegraphics[width=\textwidth]{figures/clients.jpg}
    \caption{Okno zarządzania klientami}
    \label{fig:clients}
\end{figure}

\section{Zapytanie problemowe – premia pracowników}

Okno premii pracowników umożliwia realizację zapytania problemowego
polegającego na obliczeniu wysokości premii w zależności od liczby
obsłużonych wysyłek w wybranym miesiącu i roku. Użytkownik wybiera
parametry czasowe, a następnie generuje raport prezentujący wyniki
obliczeń zgodnie z ustalonymi progami premiowymi.

\begin{figure}[H]
    \centering
    \includegraphics[width=\textwidth]{figures/bonus_window_problemquestion.jpg}
    \caption{Okno obliczania premii pracowników}
    \label{fig:bonus}
\end{figure}

\section{Analiza produktywności pracowników}

Na rysunku \ref{fig:employee_productivity} przedstawiono okno aplikacji
umożliwiające analizę produktywności pracowników na podstawie liczby
obsłużonych klientów. Funkcjonalność ta realizuje zapytanie problemowe,
którego celem jest ocena efektywności pracy personelu magazynowego.

Użytkownik może wygenerować raport, klikając przycisk
\textit{Calculate Productivity}. Wyniki prezentowane są w formie
czytelnego zestawienia tekstowego, zawierającego identyfikator
pracownika, liczbę obsłużonych klientów oraz poziom produktywności.

\begin{figure}[H]
    \centering
    \includegraphics[width=\textwidth]{figures/employeeproductivity_problemquestion.jpg}
    \caption{Okno analizy produktywności pracowników}
    \label{fig:employee_productivity}
\end{figure}

\section{Zarządzanie pracownikami}

Na rysunku \ref{fig:employees} zaprezentowano interfejs zarządzania
pracownikami systemu magazynowego. Widok ten realizuje pełny mechanizm
CRUD (Create, Read, Update, Delete) dla encji \textit{Employees}.

Górna część okna zawiera tabelę z listą pracowników, obejmującą dane
takie jak imię, nazwisko, stanowisko, data zatrudnienia oraz dane
kontaktowe. Dolna część formularza umożliwia dodawanie nowych rekordów,
edycję istniejących oraz ich usuwanie za pomocą dedykowanych przycisków.

\begin{figure}[H]
    \centering
    \includegraphics[width=\textwidth]{figures/employees.jpg}
    \caption{Interfejs zarządzania pracownikami}
    \label{fig:employees}
\end{figure}

\section{Zarządzanie stanami magazynowymi}

Widok stanów magazynowych umożliwia przegląd aktualnych ilości
produktów przypisanych do poszczególnych lokalizacji magazynowych.
Użytkownik może dodawać nowe rekordy, aktualizować istniejące dane
oraz usuwać wpisy dotyczące stanów magazynowych.

\begin{figure}[H]
    \centering
    \includegraphics[width=\textwidth]{figures/inventory.jpg}
    \caption{Widok stanów magazynowych}
    \label{fig:inventory}
\end{figure}

\section{Zarządzanie lokalizacjami magazynowymi}

Widok lokalizacji magazynowych umożliwia zarządzanie strukturą
magazynu poprzez definiowanie poszczególnych lokalizacji oraz ich
maksymalnej pojemności. Funkcjonalność ta pozwala na kontrolę
rozmieszczenia towarów oraz zapewnia poprawne działanie procesów
magazynowych.

\begin{figure}[H]
    \centering
    \includegraphics[width=\textwidth]{figures/locations.jpg}
    \caption{Widok lokalizacji magazynowych}
    \label{fig:locations}
\end{figure}

\section{Najczęściej wysyłane produkty}

Widok najczęściej wysyłanych produktów umożliwia identyfikację
produktów, których łączna ilość wysyłek przekracza zadany próg.
Użytkownik wprowadza minimalną wartość ilości, a następnie generuje
raport prezentujący produkty spełniające to kryterium.

\begin{figure}[H]
    \centering
    \includegraphics[width=\textwidth]{figures/mostshipped_problemquestion.jpg}
    \caption{Zapytanie problemowe – najczęściej wysyłane produkty}
    \label{fig:most_shipped}
\end{figure}

\section{Zarządzanie produktami}

Widok zarządzania produktami umożliwia obsługę danych dotyczących
asortymentu magazynowego. Użytkownik może dodawać nowe produkty,
modyfikować istniejące rekordy oraz usuwać dane produktów.
Informacje prezentowane są w formie tabelarycznej, co ułatwia ich
przegląd i edycję.

\begin{figure}[H]
    \centering
    \includegraphics[width=\textwidth]{figures/products.jpg}
    \caption{Interfejs zarządzania produktami}
    \label{fig:products}
\end{figure}

\section{Stopień realizacji dostawy}

Widok stopnia realizacji dostawy umożliwia obliczenie procentowego
poziomu realizacji przyjęcia towaru na podstawie ilości oczekiwanych
oraz faktycznie odebranych produktów. Użytkownik podaje identyfikator
dostawy, po czym system wykonuje odpowiednie obliczenia.

\begin{figure}[H]
    \centering
    \includegraphics[width=\textwidth]{figures/receipt_problemquestion.jpg}
    \caption{Widok obliczania stopnia realizacji dostawy}
    \label{fig:receipt_completion_input}
\end{figure}

\section{Wynik obliczenia stopnia realizacji dostawy}

Po wykonaniu obliczeń system prezentuje wynik w postaci procentowej
oraz tekstowego statusu realizacji dostawy. Informacja ta pozwala
na szybką ocenę kompletności przyjęcia towaru do magazynu.

\begin{figure}[H]
    \centering
    \includegraphics[width=\textwidth]{figures/receipt_problemquestion_result.jpg}
    \caption{Wynik obliczenia stopnia realizacji dostawy}
    \label{fig:receipt_completion_result}
\end{figure}


\section{Zarządzanie przyjęciami towaru}

Widok przyjęć towaru umożliwia rejestrowanie oraz obsługę dostaw
przyjmowanych do magazynu. Użytkownik może dodawać nowe przyjęcia,
aktualizować ich dane, usuwać wpisy oraz zarządzać szczegółami dostaw,
takimi jak produkty, ilości oczekiwane oraz faktycznie odebrane.

\begin{figure}[H]
    \centering
    \includegraphics[width=\textwidth]{figures/receipts.jpg}
    \caption{Widok zarządzania przyjęciami towaru}
    \label{fig:receipts}
\end{figure}

\section{Zarządzanie wysyłkami}

Widok wysyłek umożliwia obsługę procesów wydawania towarów z magazynu
do klientów. Użytkownik może rejestrować nowe wysyłki, przypisywać je
do klientów oraz pracowników, a także zarządzać listą produktów
przeznaczonych do wysyłki.

\begin{figure}[H]
    \centering
    \includegraphics[width=\textwidth]{figures/shipments.jpg}
    \caption{Widok zarządzania wysyłkami}
    \label{fig:shipments}
\end{figure}

\section{Zarządzanie dostawcami}

Widok zarządzania dostawcami umożliwia obsługę danych firm
współpracujących z magazynem. Użytkownik może dodawać nowych dostawców,
edytować istniejące dane oraz usuwać wpisy z bazy danych, zachowując
spójność informacji wykorzystywanych w procesach przyjęć towaru.

\begin{figure}[H]
    \centering
    \includegraphics[width=\textwidth]{figures/supplier.jpg}
    \caption{Widok zarządzania dostawcami}
    \label{fig:suppliers}
\end{figure}

\chapter{Repozytorium projektu (GitHub)}

Kod źródłowy projektu Systemu Zarządzania Magazynem został umieszczony
w publicznym repozytorium GitHub. Repozytorium zawiera pełną strukturę
projektu, w tym definicje bazy danych, skrypty SQL, kod aplikacji oraz
pliki konfiguracyjne.

Repozytorium projektu dostępne jest pod adresem:
\begin{center}
\url{https://github.com/YevhenMarchak/ProjectBazyDanych-2rok-/tree/master}
\end{center}

\section{Dokumentacja projektu (GitHub)}

Dokumentacja projektu została przygotowana w systemie \LaTeX{} i
udostępniona w publicznym repozytorium GitHub. Repozytorium zawiera
kompletną strukturę dokumentacji, w tym pliki źródłowe pracy,
konfigurację środowiska oraz materiały niezbędne do poprawnej
kompilacji dokumentu.

Dokumentacja projektu dostępna jest pod adresem:
\begin{center}
\url{https://github.com/yerzan/LatexBazyDanych}
\end{center}


% --- < bibliografia > ---
\renewcommand{\emph}[1]{\textit{#1}}
\addcontentsline{toc}{section}{\textbf{Bibliografia}}

\nocite{*}
\bibliographystyle{plain}
\bibliography{bibliografia}
\renewcommand{\emph}[1]{\uline{#1}}

% --- < spisy > ---
\clearpage
\addcontentsline{toc}{section}{\textbf{Spis rysunków}}
\listoffigures
\clearpage

%\addcontentsline{toc}{section}{\textbf{Spis tabel}}
%\listoftables
%\clearpage

\addcontentsline{toc}{section}{\textbf{Spis listingów}}
\lstlistoflistings
\clearpage

% --- < aneksy > ---
\include{appendix/statement-A}

\end{document}
