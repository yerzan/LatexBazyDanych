\chapter{Interfejs użytkownika aplikacji}

W niniejszym rozdziale przedstawiono interfejs użytkownika aplikacji
zarządzania magazynem. Zaprezentowane ekrany ilustrują najważniejsze
funkcjonalności systemu, w tym obsługę danych słownikowych, realizację
operacji CRUD oraz wizualizację zapytań problemowych.

\section{Główne okno aplikacji}

Główne okno aplikacji stanowi centralny punkt dostępu do wszystkich
funkcjonalności systemu. Użytkownik ma możliwość przechodzenia do
poszczególnych modułów, takich jak zarządzanie kategoriami, klientami,
pracownikami, produktami, lokalizacjami magazynowymi oraz realizacja
zapytań problemowych.

\begin{figure}[H]
    \centering
    \includegraphics[width=\textwidth]{figures/mainwindow.png}
    \caption{Główne okno aplikacji}
    \label{fig:mainwindow}
\end{figure}

\section{Zarządzanie kategoriami produktów}

Okno zarządzania kategoriami umożliwia wykonywanie operacji CRUD
(Create, Read, Update, Delete) na tabeli kategorii. Użytkownik może
przeglądać listę istniejących kategorii, dodawać nowe rekordy,
modyfikować istniejące dane oraz usuwać wybrane pozycje.

\begin{figure}[H]
    \centering
    \includegraphics[width=0.8\textwidth]{figures/categories.jpg}
    \caption{Okno zarządzania kategoriami}
    \label{fig:categories}
\end{figure}

\section{Zarządzanie klientami}

Moduł klientów umożliwia kompleksową obsługę danych kontrahentów.
Dostępne są operacje dodawania, edycji oraz usuwania klientów, a także
przegląd pełnej listy zapisanej w bazie danych. Dane prezentowane są
w formie tabelarycznej, co ułatwia ich analizę i wyszukiwanie.

\begin{figure}[H]
    \centering
    \includegraphics[width=\textwidth]{figures/clients.jpg}
    \caption{Okno zarządzania klientami}
    \label{fig:clients}
\end{figure}

\section{Zapytanie problemowe – premia pracowników}

Okno premii pracowników umożliwia realizację zapytania problemowego
polegającego na obliczeniu wysokości premii w zależności od liczby
obsłużonych wysyłek w wybranym miesiącu i roku. Użytkownik wybiera
parametry czasowe, a następnie generuje raport prezentujący wyniki
obliczeń zgodnie z ustalonymi progami premiowymi.

\begin{figure}[H]
    \centering
    \includegraphics[width=\textwidth]{figures/bonus_window_problemquestion.jpg}
    \caption{Okno obliczania premii pracowników}
    \label{fig:bonus}
\end{figure}

\section{Analiza produktywności pracowników}

Na rysunku \ref{fig:employee_productivity} przedstawiono okno aplikacji
umożliwiające analizę produktywności pracowników na podstawie liczby
obsłużonych klientów. Funkcjonalność ta realizuje zapytanie problemowe,
którego celem jest ocena efektywności pracy personelu magazynowego.

Użytkownik może wygenerować raport, klikając przycisk
\textit{Calculate Productivity}. Wyniki prezentowane są w formie
czytelnego zestawienia tekstowego, zawierającego identyfikator
pracownika, liczbę obsłużonych klientów oraz poziom produktywności.

\begin{figure}[H]
    \centering
    \includegraphics[width=\textwidth]{figures/employeeproductivity_problemquestion.jpg}
    \caption{Okno analizy produktywności pracowników}
    \label{fig:employee_productivity}
\end{figure}

\section{Zarządzanie pracownikami}

Na rysunku \ref{fig:employees} zaprezentowano interfejs zarządzania
pracownikami systemu magazynowego. Widok ten realizuje pełny mechanizm
CRUD (Create, Read, Update, Delete) dla encji \textit{Employees}.

Górna część okna zawiera tabelę z listą pracowników, obejmującą dane
takie jak imię, nazwisko, stanowisko, data zatrudnienia oraz dane
kontaktowe. Dolna część formularza umożliwia dodawanie nowych rekordów,
edycję istniejących oraz ich usuwanie za pomocą dedykowanych przycisków.

\begin{figure}[H]
    \centering
    \includegraphics[width=\textwidth]{figures/employees.jpg}
    \caption{Interfejs zarządzania pracownikami}
    \label{fig:employees}
\end{figure}

\section{Zarządzanie stanami magazynowymi}

Widok stanów magazynowych umożliwia przegląd aktualnych ilości
produktów przypisanych do poszczególnych lokalizacji magazynowych.
Użytkownik może dodawać nowe rekordy, aktualizować istniejące dane
oraz usuwać wpisy dotyczące stanów magazynowych.

\begin{figure}[H]
    \centering
    \includegraphics[width=\textwidth]{figures/inventory.jpg}
    \caption{Widok stanów magazynowych}
    \label{fig:inventory}
\end{figure}

\section{Zarządzanie lokalizacjami magazynowymi}

Widok lokalizacji magazynowych umożliwia zarządzanie strukturą
magazynu poprzez definiowanie poszczególnych lokalizacji oraz ich
maksymalnej pojemności. Funkcjonalność ta pozwala na kontrolę
rozmieszczenia towarów oraz zapewnia poprawne działanie procesów
magazynowych.

\begin{figure}[H]
    \centering
    \includegraphics[width=\textwidth]{figures/locations.jpg}
    \caption{Widok lokalizacji magazynowych}
    \label{fig:locations}
\end{figure}

\section{Najczęściej wysyłane produkty}

Widok najczęściej wysyłanych produktów umożliwia identyfikację
produktów, których łączna ilość wysyłek przekracza zadany próg.
Użytkownik wprowadza minimalną wartość ilości, a następnie generuje
raport prezentujący produkty spełniające to kryterium.

\begin{figure}[H]
    \centering
    \includegraphics[width=\textwidth]{figures/mostshipped_problemquestion.jpg}
    \caption{Zapytanie problemowe – najczęściej wysyłane produkty}
    \label{fig:most_shipped}
\end{figure}

\section{Zarządzanie produktami}

Widok zarządzania produktami umożliwia obsługę danych dotyczących
asortymentu magazynowego. Użytkownik może dodawać nowe produkty,
modyfikować istniejące rekordy oraz usuwać dane produktów.
Informacje prezentowane są w formie tabelarycznej, co ułatwia ich
przegląd i edycję.

\begin{figure}[H]
    \centering
    \includegraphics[width=\textwidth]{figures/products.jpg}
    \caption{Interfejs zarządzania produktami}
    \label{fig:products}
\end{figure}

\section{Stopień realizacji dostawy}

Widok stopnia realizacji dostawy umożliwia obliczenie procentowego
poziomu realizacji przyjęcia towaru na podstawie ilości oczekiwanych
oraz faktycznie odebranych produktów. Użytkownik podaje identyfikator
dostawy, po czym system wykonuje odpowiednie obliczenia.

\begin{figure}[H]
    \centering
    \includegraphics[width=\textwidth]{figures/receipt_problemquestion.jpg}
    \caption{Widok obliczania stopnia realizacji dostawy}
    \label{fig:receipt_completion_input}
\end{figure}

\section{Wynik obliczenia stopnia realizacji dostawy}

Po wykonaniu obliczeń system prezentuje wynik w postaci procentowej
oraz tekstowego statusu realizacji dostawy. Informacja ta pozwala
na szybką ocenę kompletności przyjęcia towaru do magazynu.

\begin{figure}[H]
    \centering
    \includegraphics[width=\textwidth]{figures/receipt_problemquestion_result.jpg}
    \caption{Wynik obliczenia stopnia realizacji dostawy}
    \label{fig:receipt_completion_result}
\end{figure}


\section{Zarządzanie przyjęciami towaru}

Widok przyjęć towaru umożliwia rejestrowanie oraz obsługę dostaw
przyjmowanych do magazynu. Użytkownik może dodawać nowe przyjęcia,
aktualizować ich dane, usuwać wpisy oraz zarządzać szczegółami dostaw,
takimi jak produkty, ilości oczekiwane oraz faktycznie odebrane.

\begin{figure}[H]
    \centering
    \includegraphics[width=\textwidth]{figures/receipts.jpg}
    \caption{Widok zarządzania przyjęciami towaru}
    \label{fig:receipts}
\end{figure}

\section{Zarządzanie wysyłkami}

Widok wysyłek umożliwia obsługę procesów wydawania towarów z magazynu
do klientów. Użytkownik może rejestrować nowe wysyłki, przypisywać je
do klientów oraz pracowników, a także zarządzać listą produktów
przeznaczonych do wysyłki.

\begin{figure}[H]
    \centering
    \includegraphics[width=\textwidth]{figures/shipments.jpg}
    \caption{Widok zarządzania wysyłkami}
    \label{fig:shipments}
\end{figure}

\section{Zarządzanie dostawcami}

Widok zarządzania dostawcami umożliwia obsługę danych firm
współpracujących z magazynem. Użytkownik może dodawać nowych dostawców,
edytować istniejące dane oraz usuwać wpisy z bazy danych, zachowując
spójność informacji wykorzystywanych w procesach przyjęć towaru.

\begin{figure}[H]
    \centering
    \includegraphics[width=\textwidth]{figures/supplier.jpg}
    \caption{Widok zarządzania dostawcami}
    \label{fig:suppliers}
\end{figure}
