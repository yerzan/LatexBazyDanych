\chapter{Koncepcja dostępu zdalnego do bazy danych}

\section{Założenia architektoniczne}

System Zarządzania Magazynem został zrealizowany jako aplikacja
desktopowa działająca w architekturze klient--serwer.
Aplikacja kliencka została napisana w języku Java i komunikuje się
bezpośrednio z serwerem bazy danych PostgreSQL.

Dostęp do bazy danych realizowany jest z wykorzystaniem mechanizmu
JDBC (Java Database Connectivity), który umożliwia wykonywanie
zapytań SQL oraz wywoływanie procedur i funkcji składowanych.

\section{Realizacja połączenia z bazą danych}

Połączenie aplikacji z bazą danych realizowane jest poprzez klasę
odpowiedzialną za konfigurację połączenia JDBC. Wykorzystywany jest
adres serwera, numer portu, nazwa bazy danych oraz dane użytkownika.

\section{Zakres dostępu do aplikacji}

Aplikacja została zaprojektowana jako narzędzie administracyjne
i przeznaczona jest wyłącznie do użytku przez administratora systemu.
Nie przewidziano mechanizmu obsługi wielu ról użytkowników ani
logowania końcowych użytkowników.

Administrator posiada pełny dostęp do wszystkich funkcjonalności
systemu, w tym do zarządzania danymi magazynowymi, obsługi procesów
przyjęć i wysyłek oraz wykonywania zapytań analitycznych.
Takie podejście upraszcza architekturę aplikacji oraz umożliwia
bezpośrednią kontrolę nad danymi przechowywanymi w bazie.

