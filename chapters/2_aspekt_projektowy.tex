\chapter{Aspekt projektowy baz danych}


\section{Zdefiniowane struktury tabel}

W niniejszym podrozdziale przedstawiono szczegółowy opis tabel
wchodzących w skład relacyjnej bazy danych systemu zarządzania magazynem.
Każda tabela została zaprojektowana w sposób zapewniający spójność danych
oraz poprawne odwzorowanie procesów magazynowych.

\subsection{Tabela \texttt{categories}}

Tabela \texttt{categories} przechowuje informacje o kategoriach produktów
dostępnych w magazynie.
\begin{itemize}
    \item \texttt{category\_id} (\texttt{BIGSERIAL}, PK) — unikalny identyfikator kategorii,
    \item \texttt{name} (\texttt{text}, NOT NULL) — nazwa kategorii,
    \item \texttt{description} (\texttt{text}) — opis kategorii.
\end{itemize}

\subsection{Tabela \texttt{suppliers}}

Tabela \texttt{suppliers} zawiera dane dotyczące dostawców towarów.
\begin{itemize}
    \item \texttt{supplier\_id} (\texttt{BIGSERIAL}, PK) — identyfikator dostawcy,
    \item \texttt{company\_name} (\texttt{text}, NOT NULL) — nazwa firmy dostawcy,
    \item \texttt{address} (\texttt{text}) — adres dostawcy,
    \item \texttt{phone} (\texttt{text}) — numer telefonu,
    \item \texttt{email} (\texttt{text}) — adres e-mail,
    \item \texttt{tax\_id} (\texttt{text}) — numer podatkowy.
\end{itemize}

\subsection{Tabela \texttt{clients}}

Tabela \texttt{clients} przechowuje dane klientów systemu magazynowego.
\begin{itemize}
    \item \texttt{client\_id} (\texttt{BIGSERIAL}, PK) — identyfikator klienta,
    \item \texttt{company\_name} (\texttt{text}, NOT NULL) — nazwa firmy klienta,
    \item \texttt{delivery\_address} (\texttt{text}) — adres dostawy,
    \item \texttt{phone} (\texttt{text}) — numer telefonu,
    \item \texttt{email} (\texttt{text}) — adres e-mail,
    \item \texttt{tax\_id} (\texttt{text}) — numer podatkowy.
\end{itemize}

\subsection{Tabela \texttt{employees}}

Tabela \texttt{employees} zawiera informacje o pracownikach magazynu.
\begin{itemize}
    \item \texttt{employee\_id} (\texttt{BIGSERIAL}, PK) — identyfikator pracownika,
    \item \texttt{first\_name} (\texttt{text}, NOT NULL) — imię,
    \item \texttt{last\_name} (\texttt{text}, NOT NULL) — nazwisko,
    \item \texttt{position} (\texttt{varchar(100)}, NOT NULL) — stanowisko,
    \item \texttt{hire\_date} (\texttt{date}) — data zatrudnienia,
    \item \texttt{phone} (\texttt{text}) — numer telefonu,
    \item \texttt{email} (\texttt{text}) — adres e-mail.
\end{itemize}

\subsection{Tabela \texttt{locations}}

Tabela \texttt{locations} definiuje lokalizacje magazynowe.
\begin{itemize}
    \item \texttt{location\_id} (\texttt{BIGSERIAL}, PK) — identyfikator lokalizacji,
    \item \texttt{location\_code} (\texttt{text}, NOT NULL) — kod lokalizacji,
    \item \texttt{location\_type} (\texttt{varchar(50)}, NOT NULL) — typ lokalizacji,
    \item \texttt{max\_capacity} (\texttt{numeric}) — maksymalna pojemność lokalizacji.
\end{itemize}

\subsection{Tabela \texttt{shipments}}

Tabela \texttt{shipments} rejestruje wydania towarów z magazynu do klientów.
Zawiera informacje dotyczące realizowanych wysyłek oraz osób odpowiedzialnych
za ich obsługę.

\begin{itemize}
    \item \texttt{shipment\_id} (\texttt{BIGSERIAL}, PK) — identyfikator wydania,
    \item \texttt{client\_id} (\texttt{bigint}, NOT NULL, FK $\rightarrow$ \texttt{clients.client\_id}) — klient,
    \item \texttt{employee\_id} (\texttt{bigint}, FK $\rightarrow$ \texttt{employees.employee\_id}) — pracownik realizujący wydanie,
    \item \texttt{shipment\_date} (\texttt{date}, NOT NULL) — data wydania,
    \item \texttt{status} (\texttt{varchar(50)}, NOT NULL) — status wydania
    (np. „oczekujące”, „zrealizowane”).
\end{itemize}

\subsection{Tabela \texttt{shipment\_details}}

Tabela \texttt{shipment\_details} przechowuje szczegółowe informacje dotyczące
produktów wydawanych w ramach konkretnego wydania magazynowego.
Stanowi tabelę pośrednią realizującą relację wiele do wielu pomiędzy
wydaniami a produktami.

\begin{itemize}
    \item \texttt{shipment\_id} (\texttt{bigint}, NOT NULL, FK $\rightarrow$ \texttt{shipments.shipment\_id}),
    \item \texttt{product\_id} (\texttt{bigint}, NOT NULL, FK $\rightarrow$ \texttt{products.product\_id}),
    \item \texttt{quantity} (\texttt{numeric}, NOT NULL) — ilość wydanego produktu.
\end{itemize}

\subsection{Tabela \texttt{products}}

Tabela \texttt{products} przechowuje informacje o produktach.
\begin{itemize}
    \item \texttt{product\_id} (\texttt{BIGSERIAL}, PK) — identyfikator produktu,
    \item \texttt{sku} (\texttt{text}, NOT NULL) — kod SKU produktu,
    \item \texttt{name} (\texttt{text}, NOT NULL) — nazwa produktu,
    \item \texttt{description} (\texttt{text}) — opis produktu,
    \item \texttt{category\_id} (\texttt{bigint}, FK $\rightarrow$ \texttt{categories.category\_id}) — kategoria produktu,
    \item \texttt{supplier\_id} (\texttt{bigint}, FK $\rightarrow$ \texttt{suppliers.supplier\_id}) — dostawca produktu,
    \item \texttt{weight} (\texttt{numeric}) — waga produktu,
    \item \texttt{dimensions} (\texttt{text}) — wymiary produktu.
\end{itemize}

\subsection{Tabela \texttt{inventory}}

Tabela \texttt{inventory} przechowuje informacje o stanach magazynowych.
\begin{itemize}
    \item \texttt{inventory\_id} (\texttt{BIGSERIAL}, PK) — identyfikator rekordu magazynowego,
    \item \texttt{product\_id} (\texttt{bigint}, FK $\rightarrow$ \texttt{products.product\_id}) — produkt,
    \item \texttt{location\_id} (\texttt{bigint}, FK $\rightarrow$ \texttt{locations.location\_id}) — lokalizacja magazynowa,
    \item \texttt{quantity} (\texttt{numeric}, NOT NULL) — ilość produktu,
    \item \texttt{last\_updated} (\texttt{timestamptz}) — data ostatniej aktualizacji.
\end{itemize}

\subsection{Tabela \texttt{receipts}}

Tabela \texttt{receipts} rejestruje przyjęcia towarów do magazynu.
\begin{itemize}
    \item \texttt{receipt\_id} (\texttt{BIGSERIAL}, PK) — identyfikator przyjęcia,
    \item \texttt{supplier\_id} (\texttt{bigint}, FK $\rightarrow$ \texttt{suppliers.supplier\_id}) — dostawca,
    \item \texttt{employee\_id} (\texttt{bigint}, FK $\rightarrow$ \texttt{employees.employee\_id}) — pracownik przyjmujący,
    \item \texttt{receipt\_date} (\texttt{date}, NOT NULL) — data przyjęcia,
    \item \texttt{external\_invoice\_no} (\texttt{text}) — numer faktury zewnętrznej,
    \item \texttt{status} (\texttt{varchar(50)}, NOT NULL) — status przyjęcia.
\end{itemize}

\subsection{Tabela \texttt{receipt\_details}}

Tabela \texttt{receipt\_details} zawiera szczegóły przyjęć magazynowych.
\begin{itemize}
    \item \texttt{receipt\_id} (\texttt{bigint}, FK $\rightarrow$ \texttt{receipts.receipt\_id}),
    \item \texttt{product\_id} (\texttt{bigint}, FK $\rightarrow$ \texttt{products.product\_id}),
    \item \texttt{expected\_quantity} (\texttt{numeric}, NOT NULL) — ilość oczekiwana,
    \item \texttt{received\_quantity} (\texttt{numeric}) — ilość faktycznie otrzymana.
\end{itemize}

\section{Diagram związków encji}

Struktura relacyjnej bazy danych systemu zarządzania magazynem
została przedstawiona w postaci diagramu związków encji (ERD).
Diagram obrazuje encje występujące w systemie oraz relacje
pomiędzy nimi, w tym klucze główne oraz obce.

Diagram ERD stanowi graficzne odwzorowanie struktury bazy danych
i ułatwia analizę zależności pomiędzy poszczególnymi tabelami.

\begin{figure}[H]
    \centering
    \includegraphics[width=0.95\textwidth]{figures/ERD.jpg}
    \caption{Diagram związków encji (ERD) systemu zarządzania magazynem}
    \label{fig:erd}
\end{figure}

\section{Powiązania pomiędzy tabelami}

Relacje pomiędzy tabelami w systemie zarządzania magazynem zostały
zaprojektowane w sposób odzwierciedlający rzeczywiste zależności
występujące w procesach magazynowych.
Zastosowanie kluczy obcych zapewnia integralność referencyjną danych
oraz eliminuje możliwość występowania niespójnych rekordów.

Poniżej przedstawiono najważniejsze relacje pomiędzy encjami systemu:

\begin{itemize}
    \item Relacja typu jeden do wielu (1:N) pomiędzy tabelą
    \texttt{categories} a tabelą \texttt{products}.
    Jedna kategoria może być przypisana do wielu produktów,
    natomiast każdy produkt należy do jednej kategorii.

    \item Relacja typu jeden do wielu (1:N) pomiędzy tabelą
    \texttt{suppliers} a tabelą \texttt{products}.
    Jeden dostawca może dostarczać wiele produktów,
    natomiast każdy produkt posiada jednego dostawcę.

    \item Relacja typu jeden do wielu (1:N) pomiędzy tabelą
    \texttt{locations} a tabelą \texttt{inventory}.
    Jedna lokalizacja magazynowa może zawierać wiele rekordów stanów magazynowych.

    \item Relacja typu jeden do wielu (1:N) pomiędzy tabelą
    \texttt{products} a tabelą \texttt{inventory}.
    Każdy produkt może występować w wielu lokalizacjach magazynowych.

    \item Relacja typu jeden do wielu (1:N) pomiędzy tabelą
    \texttt{suppliers} a tabelą \texttt{receipts}.
    Jeden dostawca może realizować wiele przyjęć magazynowych.

    \item Relacja typu jeden do wielu (1:N) pomiędzy tabelą
    \texttt{employees} a tabelą \texttt{receipts}.
    Jeden pracownik może obsługiwać wiele przyjęć towarów.

    \item Relacja typu wiele do wielu (N:M) pomiędzy tabelą
    \texttt{receipts} a tabelą \texttt{products},
    zrealizowana za pomocą tabeli pośredniej \texttt{receipt\_details}.
    Tabela ta przechowuje informacje o ilościach oczekiwanych oraz faktycznie
    otrzymanych produktów.

    \item Relacja typu jeden do wielu (1:N) pomiędzy tabelą
    \texttt{clients} a tabelą \texttt{shipments}.
    Jeden klient może posiadać wiele wydań magazynowych.

    \item Relacja typu jeden do wielu (1:N) pomiędzy tabelą
    \texttt{employees} a tabelą \texttt{shipments}.
    Jeden pracownik może realizować wiele wydań towarów.

    \item Relacja typu wiele do wielu (N:M) pomiędzy tabelą
    \texttt{shipments} a tabelą \texttt{products},
    zrealizowana za pomocą tabeli pośredniej \texttt{shipment\_details}.
    Tabela ta przechowuje informacje o ilości wydanych produktów.
\end{itemize}

Zaprojektowane relacje umożliwiają poprawne odwzorowanie procesów
przyjęcia oraz wydania towarów, a także zapewniają spójność
i integralność danych w całym systemie.

\section{Kod realizujący bazę danych}

W niniejszym podrozdziale przedstawiono kompletną definicję struktur
tabel relacyjnej bazy danych systemu zarządzania magazynem.
Zaprezentowane listingi zawierają instrukcje \texttt{CREATE TABLE}
odpowiedzialne za utworzenie wszystkich encji systemu wraz
z kluczami głównymi, kluczami obcymi oraz ograniczeniami integralności.

\begin{lstlisting}[language=SQL, caption={Definicja tabeli categories}]
CREATE TABLE categories (
    category_id BIGSERIAL PRIMARY KEY,
    name TEXT NOT NULL UNIQUE,
    description TEXT
);
\end{lstlisting}

\begin{lstlisting}[language=SQL, caption={Definicja tabeli suppliers}]
CREATE TABLE suppliers (
    supplier_id BIGSERIAL PRIMARY KEY,
    company_name TEXT NOT NULL,
    address TEXT,
    phone TEXT,
    email TEXT,
    tax_id TEXT
);
\end{lstlisting}

\begin{lstlisting}[language=SQL, caption={Definicja tabeli clients}]
CREATE TABLE clients (
    client_id BIGSERIAL PRIMARY KEY,
    company_name TEXT NOT NULL,
    delivery_address TEXT,
    phone TEXT,
    email TEXT,
    tax_id TEXT
);
\end{lstlisting}

\begin{lstlisting}[language=SQL, caption={Definicja tabeli employees}]
CREATE TABLE employees (
    employee_id BIGSERIAL PRIMARY KEY,
    first_name TEXT NOT NULL,
    last_name TEXT NOT NULL,
    position VARCHAR(100) NOT NULL,
    hire_date DATE,
    phone TEXT,
    email TEXT
);
\end{lstlisting}

\begin{lstlisting}[language=SQL, caption={Definicja tabeli locations}]
CREATE TABLE locations (
    location_id BIGSERIAL PRIMARY KEY,
    location_code TEXT NOT NULL UNIQUE,
    location_type VARCHAR(50) NOT NULL,
    max_capacity NUMERIC CHECK (max_capacity >= 0)
);
\end{lstlisting}

\begin{lstlisting}[language=SQL, caption={Definicja tabeli products}]
CREATE TABLE products (
    product_id BIGSERIAL PRIMARY KEY,
    sku TEXT NOT NULL UNIQUE,
    name TEXT NOT NULL,
    description TEXT,
    category_id BIGINT NOT NULL,
    supplier_id BIGINT,
    weight NUMERIC CHECK (weight >= 0),
    dimensions TEXT,
    FOREIGN KEY (category_id) REFERENCES categories(category_id),
    FOREIGN KEY (supplier_id) REFERENCES suppliers(supplier_id)
);
\end{lstlisting}

\begin{lstlisting}[language=SQL, caption={Definicja tabeli inventory}]
CREATE TABLE inventory (
    inventory_id BIGSERIAL PRIMARY KEY,
    product_id BIGINT NOT NULL,
    location_id BIGINT NOT NULL,
    quantity NUMERIC NOT NULL CHECK (quantity >= 0),
    last_updated TIMESTAMPTZ DEFAULT now(),
    UNIQUE (product_id, location_id),
    FOREIGN KEY (product_id) REFERENCES products(product_id),
    FOREIGN KEY (location_id) REFERENCES locations(location_id)
);
\end{lstlisting}

\begin{lstlisting}[language=SQL, caption={Definicja tabeli receipts}]
CREATE TABLE receipts (
    receipt_id BIGSERIAL PRIMARY KEY,
    supplier_id BIGINT NOT NULL,
    employee_id BIGINT,
    receipt_date DATE DEFAULT CURRENT_DATE,
    external_invoice_no TEXT,
    status VARCHAR(50) NOT NULL,
    FOREIGN KEY (supplier_id) REFERENCES suppliers(supplier_id),
    FOREIGN KEY (employee_id) REFERENCES employees(employee_id)
);
\end{lstlisting}

\begin{lstlisting}[language=SQL, caption={Definicja tabeli receipt\_details}]
CREATE TABLE receipt_details (
    receipt_id BIGINT NOT NULL,
    product_id BIGINT NOT NULL,
    expected_quantity NUMERIC NOT NULL CHECK (expected_quantity >= 0),
    received_quantity NUMERIC NOT NULL CHECK (received_quantity >= 0),
    PRIMARY KEY (receipt_id, product_id),
    FOREIGN KEY (receipt_id) REFERENCES receipts(receipt_id),
    FOREIGN KEY (product_id) REFERENCES products(product_id)
);
\end{lstlisting}

\begin{lstlisting}[language=SQL, caption={Definicja tabeli shipments}]
CREATE TABLE shipments (
    shipment_id BIGSERIAL PRIMARY KEY,
    client_id BIGINT NOT NULL,
    employee_id BIGINT,
    shipment_date DATE DEFAULT CURRENT_DATE,
    client_order_no TEXT,
    status VARCHAR(50) NOT NULL,
    FOREIGN KEY (client_id) REFERENCES clients(client_id),
    FOREIGN KEY (employee_id) REFERENCES employees(employee_id)
);
\end{lstlisting}

\begin{lstlisting}[language=SQL, caption={Definicja tabeli shipment\_details}]
CREATE TABLE shipment_details (
    shipment_id BIGINT NOT NULL,
    product_id BIGINT NOT NULL,
    quantity_to_ship NUMERIC NOT NULL CHECK (quantity_to_ship >= 0),
    PRIMARY KEY (shipment_id, product_id),
    FOREIGN KEY (shipment_id) REFERENCES shipments(shipment_id),
    FOREIGN KEY (product_id) REFERENCES products(product_id)
);
\end{lstlisting}

\section{Normalizacja bazy danych}

Proces normalizacji bazy danych został przeprowadzony w celu eliminacji redundancji danych,
zapewnienia spójności informacji oraz poprawy integralności logicznej systemu.
Projektowana baza danych systemu zarządzania magazynem spełnia założenia trzech
pierwszych postaci normalnych: 1NF, 2NF oraz 3NF.

\subsection{Pierwsza postać normalna (1NF)}

Pierwsza postać normalna wymaga, aby:
\begin{itemize}
    \item wszystkie atrybuty zawierały wartości atomowe,
    \item w tabelach nie występowały powtarzające się grupy danych,
    \item każdy rekord był jednoznacznie identyfikowany przez klucz główny.
\end{itemize}

W zaprojektowanej bazie danych wszystkie tabele posiadają klucze główne
(np. \texttt{category\_id}, \texttt{product\_id}, \texttt{shipment\_id}),
a każdy atrybut przechowuje pojedynczą, niepodzielną wartość.
Nie występują pola wielowartościowe ani listy danych w jednym atrybucie.

\subsection{Druga postać normalna (2NF)}

Druga postać normalna wymaga spełnienia warunków 1NF oraz braku zależności
częściowych atrybutów niekluczowych od części klucza głównego.

W bazie danych zależności częściowe zostały wyeliminowane poprzez:
\begin{itemize}
    \item rozdzielenie danych opisowych do osobnych tabel (np. \texttt{categories}, \texttt{suppliers}),
    \item zastosowanie tabel pośrednich z kluczami złożonymi, takich jak
    \texttt{receipt\_details} oraz \texttt{shipment\_details}.
\end{itemize}

W tabelach posiadających klucz złożony wszystkie atrybuty niekluczowe
zależą od całego klucza głównego, a nie od jego fragmentu.

\subsection{Trzecia postać normalna (3NF)}

Trzecia postać normalna wymaga, aby w tabelach nie występowały zależności
przechodnie pomiędzy atrybutami niekluczowymi.

W zaprojektowanej bazie danych:
\begin{itemize}
    \item dane klientów, pracowników, dostawców i produktów zostały umieszczone
    w oddzielnych tabelach,
    \item informacje zależne od innych encji są przechowywane wyłącznie poprzez
    klucze obce,
    \item nie występują atrybuty, które mogłyby być wyznaczane na podstawie innych
    atrybutów niekluczowych w tej samej tabeli.
\end{itemize}

Dzięki temu baza danych spełnia wymagania trzeciej postaci normalnej,
zapewniając wysoką spójność danych oraz łatwość dalszej rozbudowy systemu.

\subsection{Podsumowanie}

Zaprojektowana struktura bazy danych systemu zarządzania magazynem
została poprawnie znormalizowana do trzeciej postaci normalnej (3NF).
Zastosowana normalizacja minimalizuje redundancję danych, ułatwia
utrzymanie spójności informacji oraz poprawia wydajność operacji
na bazie danych.
